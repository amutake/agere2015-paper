\section{Name Uniqueness}
アクターモデルを提供する言語やライブラリ、例えば Erlang や Akka では、アクターを生成する際にプログラマが名前を指定することはしない。\footnote{Akka は付けることもできるけど}
また、アクターはシステムの進行中に動的に生成されうるものなので、動的に生成されうるアクターの名前が常に一意になることが重要である。
Actario では、Actario の意味論において、ある制限を満たした初期状態からの任意の遷移でアクターの名前が衝突しないことを証明した。

システムの遷移において満たされる、名前についての性質 trans invariant を定義し、trans invariant が確かに遷移の間で保存されることを証明、そして trans invariant が成り立てば名前の一意性が成り立つことの証明、初期状態が

trans invariant は以下の3つ

\begin{itemize}
\item[chain] システム内のすべてのアクターについて、もし自分が何らかのアクターに生成されたものならば親アクターもそのシステム内に存在する
\item[gen_fresh] システム内のすべてのアクターについて、次に生成するアクターの名前は fresh である
\item[no_dup] システム内のすべてのアクターの名前は一意である
\end{itemize}

\subsection{chain}
システム内のすべてのアクターについて、もし自分が何らかのアクターに生成されたものならば親アクターもそのシステム内に存在する、というアクターの集合についての述語をここでは chain と呼ぶ。
論理式にすると以下のようになる。(ここでは Coq の定義はあまり出したくない。なぜかというと SSReflect を使っていて説明が面倒だから)

\begin{equation}
  chain(actors : list actor) :=
  \forall a \in actors, \exists p, p = parent(a) \rightarrow p \in actors
\end{equation}

これの証明は...

\subsection{gen_fresh}
システム内のすべてのアクターについて、次に生成するアクターの名前は必ず fresh である、というアクターの集合についての述語をここでは gen_fresh と呼ぶ。
gen_fresh の定義は少し複雑になっている。次に生成するアクターの名前が必ず fresh であるということを、以下のように翻訳する。

\begin{equation}
  gen_fresh(actors : list actor) :=
  \forall a \in actors, \exists p, p = parent(a) \wedge p \in actors \rightarrow gen_number(a) < next_number(p)
\end{equation}

つまり、あるアクターについて、システム内に親アクターがいる場合、親アクターが次に生成する番号は自分の番号よりも大きい、ということにより、次に生成するアクターの名前が被らないようになっている。ただし、次に生成するアクターの名前は fresh でもその次に生成するアクターは fresh ではないこともある。例えば、同じ名前でかつ次の generation number も同じという2つのアクターがいた場合、まず片方のアクターが生成するアクターの名前は fresh だが、その次にもう片方のアクターがアクターを生成したとすると、名前が被ってしまう。また、親アクターがシステム内に存在せずに、親の親は存在しているという場合、親の親が次に生成するアクターの名前は被らないが、その子アクターが次に生成する名前は被ってしまう可能性がある。(図?)

以上のように gen_fresh だけでは gen_fresh を導けないので、gen_fresh の証明には chain と no_dup の性質が必要になる。

\begin{lstlisting}
  forall conf conf' label,
  conf ~(label)~> conf' ->
  chain conf -> gen_fresh conf -> no_dup conf ->
  gen_fresh conf'
\end{lstlisting}


\subsection{no_dup}
システム内のすべてのアクターの名前は一意であるという性質をここでは no_dup と呼ぶ。
(これが得たい性質)

以下のように表す。

\begin{equation}
  no_dup(actors : list actor) :=
  \forall a \in actors, name(a) \notin names(actors - \{a\})
  where names : Set of Actor -> Set of name
\end{equation}
