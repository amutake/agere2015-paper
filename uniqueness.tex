\section{Name Uniqueness}
アクターモデルを提供する言語やライブラリ、例えば Erlang や Akka では、アクターを生成する際にプログラマが名前を指定することはしない。\footnote{Akka は付けることもできるけど}
また、アクターはシステムの進行中に動的に生成されうるものなので、動的に生成されうるアクターの名前が常に一意になることが重要である。
Actario では、Actario の意味論において、ある制限を満たした初期状態からの任意の遷移でアクターの名前が衝突しないことを証明した。

システムの遷移において満たされる、名前についての性質 trans invariant を定義し、trans invariant が確かに遷移の間で保存されることを証明、そして trans invariant が成り立てば名前の一意性が成り立つことの証明、初期状態が

trans invariant は以下の3つ

\begin{itemize}
\item システム内のすべてのアクターについて、もし自分が何らかのアクターに生成されたものならば親アクターもそのシステム内に存在する
\item システム内のすべてのアクターについて、次に生成するアクターの名前は fresh である
\item システム内のすべてのアクターの名前は一意である
\end{itemize}

\subsection{chain}
システム内のすべてのアクターについて、もし自分が何らかのアクターに生成されたものならば親アクターもそのシステム内に存在する、という configuration についての述語をここでは chain と呼ぶ。論理式にすると以下のようになる。(ここでは Coq の定義はあまり出したくない)





\subsection{fresh}

\subsection{no dup}
