\section{Fairness}
\label{sec:fairness}

\textsf{fairness} is a property that reception of a message does not delay infinitely.
There are two variants of fairness property, weak fairness and strong fairness.
Weak fairness is that if an actor is infinitely always ready to receive the message, the message is eventually received.
Strong fairness is that if an actor is infinitely often ready to receive the message, the message is eventually received.
The Actor model satisfies strong fairness.
In this section, we define strong fairness in Actario.

\subsection{Transition Path}
Generally, fairness is represented in using operators of temporal logic.
We have to encode temporal logic because Coq does not support temporal logic.
We use transition path, which represents transition sequence of configuration, to define fairness as a predicate of transition path.
This method is used in Appl$\pi$ \cite{Affeldt200817}.

We define transition path as a function of $\mathbb{N}$ to \texttt{option config}.
In this definition, $\mathbb{N}$ represents the number of transitions from initial configuration and the reason why the return value is wrapped with \texttt{option} is that it may be no more transitions.
% つまり、i 番目の configuration から遷移先がない場合は、i + 1 番目以降は None になる。

\begin{lstlisting}
Definition path := nat -> option config.
\end{lstlisting}

And we define the predicate that the given path is correct transition path.
%% また、与えられた transition path が確かに transition path になっているか、という述語を定義する。すべての index n について、n 番目の configuration が存在するならば、n + 1 番目の configuration が存在するならそれは遷移できるものか、それ以上遷移できない。n 番目の configuration が存在しないならば、その次の configuration も存在しない、という意味である。

\begin{lstlisting}
Definition is_transition_path (p : path) : Prop :=
  forall n,
    (forall c, p n = Some c ->
               (exists c' l, p (S n) = Some c' /\ c ~(l)~> c') \/
               p (S n) = None) /\
    (p n = None -> p (S n) = None).
\end{lstlisting}

\subsection{enabled}
We define the predicate that the transition from the given configuration with the given label is possible, called \texttt{enabled}.
\texttt{enabled} is defined as there exists a configuration after transitioning from the configuration with the label.
In Actario, \texttt{enabled} is defined as follows.
%% ある遷移ができる状態にある、ということを enabed と呼ぶ。これは、ある configuration からあるラベルによって遷移した先の configuration が存在する、と定義する。

\begin{lstlisting}
Definition enabled (c : config) (l : label) : Prop :=
  exists c', c ~(l)~> c'.
\end{lstlisting}

\subsection{infinitely often enabled}
We define the predicate that the transition is infinitely often enabled in the transition path.
%% これは、すべての index n について、n 番目の configuration があるラベルによって遷移が可能ならば、その先そのラベルによって遷移が可能になる configuration が存在する、と定義する。

\begin{lstlisting}
Definition infinitely_often_enabled (l : label) (p : path) : Prop :=
  forall n c, p n = Some c ->
              enabled c l ->
              exists m c', m > n /\
                           p m = Some c' /\
                           enabled c' l.
\end{lstlisting}


\subsection{eventually processed}
We define \texttt{eventually processed} that is the predicate of label and transition path.
It represents that the transition with the label is processed eventually in the path.
It is defined as follows.

\begin{lstlisting}
Definition eventually_processed (l : label) (p : path) : Prop :=
  exists n c c',
    p n = Some c /\ p (S n) = Some c' /\ c ~(l)~> c'.
\end{lstlisting}


\subsection{Definition of fairness}
Then we can define \texttt{fairness} predicate for transition path.
For the given transition path and for each label, if \texttt{infinitely often enabled} holds, then \texttt{eventually processed} holds.
\texttt{is postfix of} predicate is used for representing 'infinite'.
If \texttt{is postfix of} is not used, the transition may not be processed after the transition is processed although the transition is processed in whole the path.
To prevent it, if \texttt{inifinitely often enabled} holds then \texttt{eventually processed} holds for arbitrary postfix path by using \texttt{is postfix path}.

\begin{lstlisting}
Definition is_postfix_of (p' p : path) : Prop :=
  exists n, (forall m, p' m = p (m + n)).

Definition fairness : Prop :=
  forall p p', is_postfix_of p' p ->
               (forall l,
                  infinitely_often_enabled l p' ->
                  eventually_processed l p').
\end{lstlisting}
