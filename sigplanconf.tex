%-----------------------------------------------------------------------------
%
%               Template for sigplanconf LaTeX Class
%
% Name:         sigplanconf-template.tex
%
% Purpose:      A template for sigplanconf.cls, which is a LaTeX 2e class
%               file for SIGPLAN conference proceedings.
%
% Guide:        Refer to "Author's Guide to the ACM SIGPLAN Class,"
%               sigplanconf-guide.pdf
%
% Author:       Paul C. Anagnostopoulos
%               Windfall Software
%               978 371-2316
%               paul@windfall.com
%
% Created:      15 February 2005
%
%-----------------------------------------------------------------------------


\documentclass[10pt]{sigplanconf}

% The following \documentclass options may be useful:

% preprint      Remove this option only once the paper is in final form.
% 10pt          To set in 10-point type instead of 9-point.
% 11pt          To set in 11-point type instead of 9-point.
% authoryear    To obtain author/year citation style instead of numeric.

\usepackage{amssymb}
\usepackage{amsmath}
\usepackage{xcolor}
\usepackage{listings,lstcoq,lsterlang}
\usepackage{bcprules}
\usepackage{url}
\usepackage{enumitem}

\newtheorem{definition}{Definition}
\newtheorem{theorem}{Theorem}
\newtheorem{lemma}{Lemma}


\lstdefinestyle{default}{
  language=coq,
  basicstyle={\ttfamily\normalsize},
  breaklines=true,
  frame=tb,
  framesep=6pt,
  captionpos=b
}
\lstdefinestyle{small}{
  style=default,
  basicstyle={\ttfamily\small},
}
\lstset{
  style=default
}

\begin{document}

\special{papersize=8.5in,11in}
\setlength{\pdfpageheight}{\paperheight}
\setlength{\pdfpagewidth}{\paperwidth}

\conferenceinfo{AGERE!@SPLASH}{Oct., 2015, Pittsburgh, PA, USA}
\copyrightyear{20yy}
\copyrightdata{978-1-nnnn-nnnn-n/yy/mm}
\doi{nnnnnnn.nnnnnnn}

% Uncomment one of the following two, if you are not going for the
% traditional copyright transfer agreement.

%\exclusivelicense                % ACM gets exclusive license to publish,
                                  % you retain copyright

%\permissiontopublish             % ACM gets nonexclusive license to publish
                                  % (paid open-access papers,
                                  % short abstracts)

%% \titlebanner{banner above paper title}        % These are ignored unless
%% \preprintfooter{short description of paper}   % 'preprint' option specified.

\title{Actario: A Framework for Reasoning About Actor Systems}
%% \subtitle{Subtitle Text, if any}

\authorinfo{Shohei Yasutake}
           {Tokyo Institute of Technology}
           {yasutake@psg.cs.titech.ac.jp}
\authorinfo{Takuo Watanabe}
           {Tokyo Institute of Technology}
           {takuo@acm.org}

\maketitle

\begin{abstract}
The two main characteristics of the Actor model are asynchronous message passing and dynamic system topology. The latter relies on the on-the-fly creation of actor names that often complicates the formal treatment of systems described in the Actor model. In this paper, we introduce Actario, a formalization of the Actor model in Coq. Actario incorporates a name creation mechanism that is formally proven to generate a consistent set of actor names. The mechanism helps proper handling of names in modeling and reasoning about actor-based systems. Actario also provides a code extraction mechanism that generates Erlang programs.
\end{abstract}

\category{F.3.2}{Logics and Meanings of Programs}%
{Specifying and Verifying and Reasoning about Programs}%
[Mechanical verification]

% general terms are not compulsory anymore,
% you may leave them out
\terms{Actors, Formal Models}

\keywords{Actor Model, Formalization, Actario, Coq, Erlang}

\section{Introduction}
アクターモデルは並行計算のモデルの一つである\cite{Agha:1986aa}。
アクターモデルではアクターと呼ばれる計算実体が非同期にメッセージを送り合うことで計算を進める。
あるアクターがメッセージに対してどういう動作を行うか、ということをそのアクターの振る舞いという。
アクターはメッセージを受け取ると、新しいアクターを作る、他のアクターにメッセージを送る、自分自身の振る舞いを変える、という3つの動作を行うことができる。

アクターモデルに基づく言語は多数提案・実装されてきており、それらの研究成果をもとに現在 Erlang、Akka 等、アクターモデルを並行計算の基盤としたプログラミング言語やライブラリが実用に供されている。
そのため、アクターモデルに寄って構成されたシステムの形式的検証は喫緊の課題であると考えられる。

形式的検証については、アクターで記述されたシステムのモデル検査を可能にするモデル記述言語 Rebeca、
証明支援系 Athena を用いた形式化、
および Coq を用いたアクターモデルの定式化など、
最近になっていくつかの研究成果が出ている。
本論文では、Actario という Coq 上でアクターモデルの形式化を提案する。
Actario は \url{https://github.com/amutake/actario} でソースコードが公開されている。

% 形式的検証をおこなう上での問題
% 特にアクターの名前の生成,一貫性

アクターの名前は必ず一意である必要がある。
アクターはシステムの進行に伴って動的に生成されるが、その中で fresh でない名前の生成が行われることはない。
fresh なアクターの名前の付け方には、~がある。

% ここで解決したい問題は何か(問題)
% Actarioではどのようにして問題を解決しようとしているか(手法)
% 提案手法
% * 名前付けの方法とそれによる名前の一貫性の形式的証明
% それによってどのようなことが明らかになるか:
% * 普通のプログラミングと同様にモデル化できる
% * 名前を勝手につけられるので検証は難しくなる(?)

% Actarioの特徴
% * 普通のプログラミングと同様にシステムを記述できる
% * Coqによって検証もできる
%   + 実装しているシステム自身はCoqによって検証されている
%     - 名前の一貫性(済)
%     - アクターのpersistence (済)
%     - メッセージのpersistence (済)
% * Erlangプログラムを生成できる

2節では Actario で使うアクターのシンタックスを説明する。
3節では Actario で用いるアクターモデルの意味論について説明する。
4節では、Actario が用いる意味論において、動的に生成されるアクターの名前が一意になるという定理の証明を説明する。
5節では Actario における fairness の定式化について説明し、
6節で他のアクターモデルの形式化や並行・分散システムの検証用フレームワークとの差異を述べる。

\newpage % oveview のタイトルだけが前ページに残ったため
\section{Overview of Actario}
\label{sec:overview}

\subsection{Programming in Actario}

Actario is a Coq framework for defining and verifying actor-based
systems. A typical workflow using Actario is as follows.
\begin{enumerate}
\item Describe an actor-based system using types and notations defined in the framework.
\item Specify and verify desired properties of the system.
\item Extract the Erlang version of the system using the code extraction mechanism of Coq.
\end{enumerate}

Actario is currently under development and still does not provide
convenient libraries of predicates, lemmas, tactics and so
forth. Thus, verifying a user-defined actor system may involve a large
amount of work. However, we have formally proved that the underlying
execution model provided in the framework satisfies three important
actor properties: name uniqueness, actor persistence and name
persistence. The fact implies that a system described using Actario is
guaranteed to have these actor properties.

\paragraph{Example: Recursive Factorial System}
Let us use a simple example to illustrate the usage of Actario.
Figure~\ref{coq:fact} is the definition of an actor system that implements the
continuation-passing style factorial function adapted from
\cite{Agha:1986aa}.  Two recursive functions \lstinline|factorial_behv| and
\lstinline|factorial_cont_behv| define the behaviors of factorial actors and
continuation actors respectively.  The function \lstinline|factorial_system| the
system by being applied to a natural number and the name of a customer
actor that is intended to receive the final result.  In this function,
a factorial actor is created and bound to the variable \lstinline|x|, then a pair
of the number and customer is sent to the actor.

\begin{figure}[t]
\begin{lstlisting}[style=small]
Definition factorial_cont_behv (val : nat)
                               (cust : name) :=
  receive (fun msg =>
    match msg with
     | nat_msg arg => cust ! nat_msg (val * arg);
                      become empty_behv
     | _ => become empty_behv
    end).

CoFixpoint factorial_behv :=
  receive (fun msg =>
    match msg with
     | tuple_msg (nat_msg 0) (name_msg cust) =>
       cust ! nat_msg 1;
       become factorial_behv
     | tuple_msg (nat_msg (S n)) (name_msg cust) =>
       cont <- new (factorial_cont_behv (S n) cust);
       me <- self;
       me ! tuple_msg (nat_msg n) (name_msg cont);
       become factorial_behv
     | _ => become factorial_behv
    end).

Definition factorial_system (n : nat) (cust : name) :=
  init "factorial" (
    x <- new factorial_behv;
    x ! tuple_msg (nat_msg n) (name_msg cust);
    become empty_behv
  ).
\end{lstlisting}
\caption{Recursive Factorial System in Actario}\label{coq:fact}
\end{figure}

%% \begin{figure}[t]
%% \centering
%% \includegraphics[width=8cm]{./images/fact.pdf}
%% \caption{Recursive Factorial}\label{fig:fact}
%% \end{figure}


\subsection{Types and Notations}

\subsubsection{Messages}
アクター間で取り交わされるメッセージの型 \texttt{message} は図\ref{coq:message}と定義されている。
空のメッセージ、アクターの名前のメッセージ、基本的な型 (自然数、文字列、ブール値) のメッセージ、タプルのメッセージを扱うことができる。

\begin{figure}[tb]
\begin{lstlisting}
Inductive message : Set :=
 | empty_msg : message
 | name_msg : name -> message
 | str_msg : string -> message
 | nat_msg : nat -> message
 | bool_msg : bool -> message
 | tuple_msg : message -> message -> message.
\end{lstlisting}
\caption{Message Type}\label{coq:message}
\end{figure}


\subsubsection{Actions and Behaviors}

\texttt{actions} はアクターが行うアクションの列を表す型である。
アクションの種類は、他のアクターに向けてメッセージを送る\texttt{send}、新しいアクターを作る\texttt{new}、自分自身の名前を得る\texttt{self}、次のメッセージに対する振る舞いを決める\texttt{become}があり、これらが \texttt{actions} の値コンストラクタとなっている。
behavior is defined as a function from message to actions.

\begin{figure}[tb]
\begin{lstlisting}
CoInductive actions : Type :=
 | new : behavior -> (name -> actions) -> actions
 | send : name -> message -> actions -> actions
 | self : (name -> actions) -> actions
 | become : behavior -> actions
with behavior : Type :=
 | receive : (message -> actions) -> behavior.
\end{lstlisting}
\caption{Types for Actions and Behaviors}\label{coq:actions}
\end{figure}

\texttt{actions} および \texttt{behavior} は Actario では図\ref{coq:actions} のように定義されている。
\texttt{actions} のそれぞれのコンストラクタを説明する。

\begin{list}{}{%
%    \setlength{\labelwidth}{0pt}
%    \setlength{\labelsep}{0pt}
    \setlength{\leftmargin}{1.5em}
    \setlength{\itemindent}{-1.5em}
}
\item \lstinline|new : behavior -> (name -> actions) -> actions|\\
  新しいアクターを生成するアクションである。
  引数として与えられたある behavior を initial behavior として新しいアクターを生成し、その名前をアクションの継続に渡す。
\item \lstinline|send : name -> message -> actions -> actions|\\
  指定した名前のアクターにメッセージを送るアクションである。
  第一引数に指定した名前のアクターに、第二引数で指定したメッセージを送り、第三引数で指定したその後のアクションを実行する、という意味である。
\item \lstinline|self : (name -> actions) -> actions| \\
  自分自身の名前を得るアクションである。
  自分の名前を第一引数で与えられた継続に渡し、残りのアクションを得る。
\item \lstinline|become : behavior -> actions| \\
  次のメッセージに対して指定した振る舞いで処理する。
  また、次のメッセージの待ち状態になることも表し、これ以降アクションは取れないようになっている。
\end{list}

actions と behavior は co-inductive types として定義される。
これは become で現在の behavior を指定したり、ある他の behavior と相互再帰するようなことがあるからである。
Coq ではこのような、コンストラクタの深さが小さくならない再帰は通常の inductive definition では書けなくなっている。

\subsubsection{Notations}

また、アクターモデルを採用しているプログラミング言語に近づけるように図\ref{coq:notation}のような notation をつけている。
\lstinline|n <- new behv; cont| は \lstinline|new behv (fun n => cont)| に変換され、
\lstinline|n ! msg; cont| は \lstinline|send n msg cont| に変換され、
\lstinline|me <- self; cont| は \lstinline|self (fun me => cont)| に変換される。

\begin{figure}[tb]
\begin{lstlisting}
Notation ``n '<-' 'new' behv ; cont'' :=
    (new behv (fun n => cont))
    (at level 0, cont at level 10).
Notation ``n '!' m ';' a'' :=
    (send n m a) (at level 0, a at level 10).
Notation ``me '<-' 'self' ';' cont'' :=
    (self (fun me => cont))
    (at level 0, cont at level 10).
\end{lstlisting}
\caption{Notations for Actions}\label{coq:notation}
\end{figure}

\section{semantics}

操作的意味論を labeled transition system として形式化する。
ラベルとして receive, send, new, self の4つを用意する。

\begin{lstlisting}
  Inductive label :=
  | Receive (to : name) (from : name) (content : message) (* `to` receives a message `content` from `from` *)
  | Send (from : name) (to : name) (content : message)    (* `from` sends a message `content` to `to` *)
  | New (child : name)                                    (* an actor named `child` is created *)
  | Self (me : name).                                     (* `me` gets own name *)
\end{lstlisting}

\texttt{Receive (to : name) (from : name) (content : message)} は to という名前のアクターが from という名前のアクターからのメッセージ content を受け取ったということを表す。
\texttt{Send (from : name) (to : name) (content : message)} は from という名前のアクターが to という名前のアクターに向けてメッセージ content を送ったということを表す。
\texttt{New (child : name)} は child という名前の新しいアクターを生成したということを表し、
\texttt{Self (me : name)} は me という名前のアクターが自分自身の名前を読みだしたということを表す。

このラベル一つ一つに対応した意味論は図\ref{semantics}にある。
receive はメッセージ待ち状態にあるアクターが、自身に向けて送られたメッセージを受け取り、アクションの列を生成する遷移である。
send はあるアクターが他のアクターに向けてメッセージを送る遷移。
...
これを Coq 上で定義すると図\ref{coq:semantics}のようになる。ほとんどそのままだが集合の代わりにリストを使っている。なぜならば...

\begin{figure}[h]
  \infrule[Receive]{
  }{
    sendings \uplus \{Sending(to, from, content)\} \bowtie actors \cup \{Actor(to, become(behv), next)\} \leadsto^{Receive(to, from, content)}
    sendings \bowtie actors \cup \{Actor(to, behv(content), next)\}
  }
  \infrule[Send]{
  }{
    sendings \bowtie actors \cup \{Actor(from, to \ ! \ content; cont, next)\} \leadsto^{Send(from, to, content)}
    sendings \uplus \{Sending(to, from, content)\} \bowtie actors \cup \{Actor(from, cont, next)\}
  }
  \infrule[New]{
  }{
    sendings \bowtie actors \cup \{Actor(parent, n \leftarrow new behv; cont, next)\} \leadsto^{New(child)}
    sendings \bowtie actors \cup \{Actor(parent, cont{n/child}, next + 1), Actor(child, become(behv), 0)\}
    where child := Generated(next, parent)
  }
  \infrule[Self]{
  }{
    sendings \bowtie actors \cup \{Actor(me, n \leftarrow self; cont, next)\} \leadsto^{Self(me)}
    sendings \bowtie actors \cup \{Actor(me, cont{n/me}, next)\}
  }
\end{figure}




\begin{lstlisting}
Reserved Notation "c1 '~(' t ')~>' c2" (at level 60).
Inductive trans : label -> config -> config -> Prop :=
(* receive transition *)
| trans_receive :
    forall to from content f gen sendings_l sendings_r actors_l actors_r,
      (sendings_l ++ Build_sending to from content :: sendings_r)
                 >< (actors_l ++ Build_actor to (become (receive f)) gen :: actors_r)
        ~(Receive to from content)~>
        (sendings_l ++ sendings_r) >< (actors_l ++ Build_actor to (f content) gen :: actors_r)
(* send transition *)
| trans_send :
    forall from to content cont gen sendings_l sendings_r actors_l actors_r,
      (sendings_l ++ sendings_r)
                 >< (actors_l ++ Build_actor from (send to content cont) gen :: actors_r)
         ~(Send from to content)~>
         (sendings_l ++ Build_sending to from content :: sendings_r)
           >< (actors_l ++ Build_actor from cont gen :: actors_r)
(* new transition *)
| trans_new :
    forall parent behv cont gen sendings actors_l actors_r,
      sendings >< (actors_l ++ Build_actor parent (new behv cont) gen :: actors_r)
        ~(New (generated gen parent))~>
        sendings ><
          (Build_actor (generated gen parent) (become behv) 0 ::
          actors_l ++
          Build_actor parent (cont (generated gen parent)) (S gen) ::
          actors_r)
(* self transition *)
| trans_self :
    forall me cont gen sendings actors_l actors_r,
      sendings >< (actors_l ++ Build_actor me (self cont) gen :: actors_r)
        ~(Self me)~>
        sendings >< (actors_l ++ Build_actor me (cont me) gen :: actors_r)
where "c1 '~(' t ')~>' c2" := (trans t c1 c2).
\end{lstlisting}

\section{Name Uniqueness}
\label{sec:uniqueness}
%% \note{なぜこれが証明されていなければいけないのか、これが証明されていることによってどういうことが言えるのか}
In programming languages or libraries providing the Actor model such as Erlang or Akka,
the system automatically generates actors with fresh names without specifying the name explicitly by the programmer.
In Actario, the proposition that all actor names in the configuration are not duplicate by any transitions is proven.

%% アクターモデルを提供する言語やライブラリ、例えば Erlang や Akka では、アクターを生成する際にプログラマが名前を指定せずとも fresh な名前を生成してくれる。
%% また、アクターはシステムの進行中に動的に生成されうるものなので、動的に生成されうるアクターの名前が常に一意になることが重要である。
%% Actario では、Actario の意味論において、ある制限を満たした初期状態からの任意の遷移でアクターの名前が衝突しないことを証明した。

To prove, we define an invariant about actor names preserved between any transitions. It is named \textit{trans invariant}.
The trans invariant consists of the following three predicates for configuration.
%% システムの遷移において満たされる、名前についての性質 trans invariant を定義し、trans invariant が確かに遷移の間で保存されることを証明、そして trans invariant が成り立てば名前の一意性が成り立つことの証明、初期状態が

\begin{displaymath}
  \begin{array}{l}
    \texttt{trans\_invariant}(c) := \\
    \quad \texttt{chain}(c) \wedge \texttt{gen\_fresh}(c) \wedge \texttt{no\_dup}(c)
  \end{array}
\end{displaymath}

The brief explanations of \texttt{chain}, \texttt{gen\_fresh}, and \texttt{no\_dup} are followings:

\begin{description}[style=nextline,leftmargin=12pt,parsep=0pt]
\item[\texttt{chain}]
  For each actor in the configuration, if the actor is generated by another actor, then the parent actor is also in the configuration.
\item[\texttt{gen\_fresh}]
  For each actor in the configuration, actor name genereted by the actor in the next is fresh.
\item[\texttt{no\_dup}]
  For all actor name in the configuration are unique.
\end{description}

\subsection{functions}

Before starting the explanation and the proof, we define some functions used in this section.

\begin{description}[style=nextline,leftmargin=12pt,parsep=0pt]
\item[\texttt{actors} $: \textit{Configuration} \rightarrow \textit{Set(Actor)}$]
  \texttt{actors} returns the set of actors in the given configuration.
\item[\texttt{parent} $: \textit{Actor} \rightarrow \textit{Actor}$]
  \texttt{parent} returns the parent actor of the given actor.
  If the given actor is toplevel actor, the function returns nothing. % null?
\item[\texttt{gen\_number} $: \textit{Actor} \rightarrow \mathbb{N}$]
  \texttt{gen\_number} returns generated number of the name of the given actor.
  If the given actor is toplevel actor, the function returns nothing.
\item[\texttt{next\_number} $: \textit{Actor} \rightarrow \mathbb{N}$]
  \texttt{next\_number} returns next generation number of the given actor.
\item[\texttt{name} $: \textit{Actor} \rightarrow \textit{Name}$]
  \texttt{name} returns the name of the given actor.
\item[\texttt{names} $: \textit{Set(Actor)} \rightarrow \textit{Set(Name)}$]
  \texttt{names} returns names of the given set of actors.
\end{description}

\subsection{chain}
We define a predicate of configuration, called \texttt{chain}.
\texttt{chain} is the predicate that, for each actor in the given configuration, if it is generated by another actor, the parent actor is also in the configuration.
\texttt{chain} is defined as the following.

\begin{displaymath}
  \begin{array}{l}
    \texttt{chain}(c) := \\
    \quad \forall a \in \texttt{actors}(c), \forall p, p = \texttt{parent}(a) \Rightarrow p \in \texttt{actors}(c)
  \end{array}
\end{displaymath}

Then, we can prove \textit{chain preservation property} that chain is preserved between any transitions.
The proof is by case analysis on the label.
\texttt{chain} is decided by only actor names, and the transition which have a possibility to change the names in the configuration is only \textsc{New} transition.
Therefore, we consider only the case of \textsc{New} transition.

\begin{lemma}{chain preservation}
\begin{displaymath}
  \begin{array}{l}
    \forall c, c' \in \textit{Configuration}, \forall l \in \textit{Label}, \\
    \quad \texttt{chain}(c) \wedge c \overset{l}{\leadsto} c' \Rightarrow \texttt{chain}(c')
  \end{array}
\end{displaymath}
\end{lemma}

\subsection{gen\_fresh}
We define \texttt{gen\_fresh} predicate that, for each actor in the configuration, the name of its child is always fresh.
The definition of \texttt{gen\_fresh} is complicated a little.
We translate the proposition that next generated name is fresh to the following.

\begin{displaymath}
  \begin{array}{l}
    \texttt{gen\_fresh}(c) := \\
    \quad \forall a \in \texttt{actors}(c), \forall p \in \texttt{actors}(c), p = \texttt{parent}(a) \Rightarrow \\
    \quad \quad \quad \texttt{gen\_number}(a) < \texttt{next\_number}(p)
  \end{array}
\end{displaymath}


It is guaranteed that the actor name generated in the next is fresh if satisfying \texttt{gen\_fresh} predicate by the relation of next generation numbers and actor names. %% For each actor in the configuration, if its parent is in the configuration, the next generation number of the parent actor is greater than the generation number of the name of the child actor.
However, the actor name generated after the next is not always fresh name.
For example, if there are two actors ($A$ and $B$) that have the same name and the same next generation number and actor $A$ generates a child actor and actor $B$ generates a child actor, although \texttt{gen\_fresh} holds, these child actors have the same name.
Furthermore, if the parent of the actor $A$ does not exist in the configuration and the parent of the parent exists in the configuration, and the parent of the parent actor generates an actor and it also generates an actor, then the name is possible to have the same as $A$'s one.

%% つまり、あるアクターについて、システム内に親アクターがいる場合、親アクターが次に生成する番号は自分の番号よりも大きい、ということにより、次に生成するアクターの名前が被らないようになっている。
%% ただし、次に生成するアクターの名前は fresh でもその次に生成するアクターは fresh ではないこともある。
%% 例えば、同じ名前でかつ次の generation number も同じという2つのアクターがいた場合、まず片方のアクターが生成するアクターの名前は fresh だが、その次にもう片方のアクターがアクターを生成したとすると、名前が被ってしまう。
%% また、親アクターがシステム内に存在せずに、親の親は存在しているという場合、親の親が次に生成するアクターの名前は被らないが、その子アクターが次に生成する名前は被ってしまう可能性がある。(図?)

Thus, to prove \textit{gen fresh preservation} proposition that \texttt{gen\_fresh} is preserved between transitions, it is necessary to use \texttt{chain} and \texttt{no\_dup} as hypotheses.
%% The proof is by ...
%% 以上のように gen\_fresh だけでは gen\_fresh を導けないので、gen\_fresh の証明には chain と no\_dup の性質が必要になる。

\begin{lemma}{gen fresh preservation}
\begin{displaymath}
  \begin{array}{l}
    \forall c, c' \in \textit{Configuration}, \forall l \in \textit{Label}, \\
    \quad \texttt{chain}(c) \wedge \texttt{gen\_fresh}(c) \wedge \texttt{no\_dup}(c) \wedge c \overset{l}{\leadsto} c' \Rightarrow \\
    \quad \texttt{gen\_fresh}(c')
  \end{array}
\end{displaymath}
\end{lemma}

\subsection{no\_dup}
We define \texttt{no\_dup} predicate that all actor names in the given configuration are unique.
This is the property we have to prove.
\texttt{no\_dup} is defined as the following.

\begin{displaymath}
  \begin{array}{l}
    \texttt{no\_dup}(c) := \\
    \quad \forall a \in \texttt{actors}(c), \texttt{name}(a) \notin
    \texttt{names}(\texttt{actors}(c) \setminus \{a\})
  \end{array}
\end{displaymath}

We proved \textit{no dup preservation} property defined as the following.
It represents that if the actor names in the configuration is not duplicate and the next generated actor name is fresh, then \texttt{no\_dup} holds in the next configuration.

\begin{lemma}{no dup preservation}
\begin{displaymath}
  \begin{array}{l}
    \forall c, c' \in \textit{Configuration}, \forall l \in \textit{Label}, \\
    \quad \texttt{gen\_fresh}(c) \wedge \texttt{no\_dup}(c) \wedge c \overset{l}{\leadsto} c' \Rightarrow \texttt{no\_dup}(c')
  \end{array}
\end{displaymath}
\end{lemma}

\subsection{uniqueness}
Then, we start to prove name uniqueness.
First, we prove trans invariant preservation that trans invariant is preserved between transitions.
This is obvious by chain preservation, gen fresh preservation and no dup preservation.
\begin{lemma}{trans invariant preservation}
  \begin{displaymath}
    \begin{array}{l}
      \forall c, c' \in \textit{Configuration}, \forall l \in \textit{Label}, \\
      \quad \texttt{trans\_invariant}(c) \wedge c \overset{l}{\leadsto} c' \Rightarrow \\
      \quad \texttt{trans\_invariant}(c')
    \end{array}
  \end{displaymath}
\end{lemma}

Next, we prove that if trans invariant holds in initial configuration, trans invariant holds after arbitrary transitions.

%% 次に初期状態について trans\_invariant が成り立っていれば、任意回の遷移後も trans\_invariant が成り立つということをを証明する。

\begin{lemma}{trans invariant preservation star}
  \begin{displaymath}
    \begin{array}{l}
      \forall c, c' \in \textit{Configuration}, \forall l \in \textit{Label}, \\
      \quad \texttt{trans\_invariant}(c) \wedge c \overset{l}{\leadsto\star} c' \Rightarrow \\
      \quad \texttt{trans\_invariant}(c')
    \end{array}
  \end{displaymath}
\end{lemma}
$c \overset{l}{\leadsto\star} c'$ represents reflexive transitive closure of transition.
The proof is by induction of reflexive transitive closure of transition and trans invariant preservation.

Finally, we can prove name uniqueness.
\begin{theorem}{name uniqueness}
  \begin{displaymath}
    \begin{array}{l}
      \forall c, c' \in \textit{Configuration}, \forall l \in \textit{Label}, \\
      \quad \texttt{trans\_invariant}(c) \wedge c \overset{l}{\leadsto\star} c' \Rightarrow \texttt{no\_dup}(c')
    \end{array}
  \end{displaymath}
\end{theorem}
This is obvious by trans invariant preservation star because \texttt{no\_dup} is in \texttt{trans\_invariant}.

\section{Fairness}
\label{sec:fairness}

\textsf{fairness} is a property that reception of a message does not delay infinitely.
There are two variants of fairness property, weak fairness and strong fairness.
Weak fairness is that if an actor is infinitely always ready to receive the message, the message is eventually received.
Strong fairness is that if an actor is infinitely often ready to receive the message, the message is eventually received.
The Actor model satisfies strong fairness.
We have not proved any properties using strong fairness yet, but for a case study, we explain how to define strong fairness in Actario.

\subsection{Transition Path}
Generally, fairness is represented in using operators of temporal logic.
We have to encode temporal logic because Coq does not support temporal logic.
We use transition path, which represents transition sequence of configuration, to define fairness as a predicate of transition path.
This method is used in Appl$\pi$ \cite{Affeldt200817}.

We define transition path as a function of $\mathbb{N}$ to \texttt{option config}.
In this definition, $\mathbb{N}$ represents the number of transitions from initial configuration and the reason why the return value is wrapped with \texttt{option} is that it may be no more transitions.
% つまり、i 番目の configuration から遷移先がない場合は、i + 1 番目以降は None になる。

\begin{lstlisting}
Definition path := nat -> option config.
\end{lstlisting}

And we define the predicate that the given path is correct transition path.
%% また、与えられた transition path が確かに transition path になっているか、という述語を定義する。すべての index n について、n 番目の configuration が存在するならば、n + 1 番目の configuration が存在するならそれは遷移できるものか、それ以上遷移できない。n 番目の configuration が存在しないならば、その次の configuration も存在しない、という意味である。

\begin{lstlisting}
Definition is_transition_path
    (p : path) : Prop :=
  forall n,
    (forall c, p n = Some c ->
      (exists c' l, p (S n) = Some c' /\
        c ~(l)~> c') \/
      p (S n) = None) /\
    (p n = None -> p (S n) = None).
\end{lstlisting}

\subsection{Enabled}
We define the predicate that the transition from the given configuration with the given label is possible, called \texttt{enabled}.
In Actario, \texttt{enabled} is defined as there exists a configuration after transitioning from the configuration with the label, as follows.
%% ある遷移ができる状態にある、ということを enabed と呼ぶ。これは、ある configuration からあるラベルによって遷移した先の configuration が存在する、と定義する。

\begin{lstlisting}
Definition enabled (c : config)
    (l : label) : Prop :=
  exists c', c ~(l)~> c'.
\end{lstlisting}

\subsection{Infinitely Often Enabled}
We define the predicate that the transition is infinitely often enabled in the transition path.
It is named \texttt{infinitely often enabled}.
%% これは、すべての index n について、n 番目の configuration があるラベルによって遷移が可能ならば、その先そのラベルによって遷移が可能になる configuration が存在する、と定義する。

\begin{lstlisting}
Definition infinitely_often_enabled
    (l : label) (p : path) : Prop :=
  forall n c, p n = Some c ->
    enabled c l ->
    exists m c', m > n /\
      p m = Some c' /\
      enabled c' l.
\end{lstlisting}


\subsection{Eventually Processed}
We define \texttt{eventually processed} that is the predicate of label and transition path.
It represents that the transition with the label is processed eventually in the path.
It is defined as follows.

\begin{lstlisting}
Definition eventually_processed
    (l : label) (p : path) : Prop :=
  exists n c c',
    p n = Some c /\
    p (S n) = Some c' /\
    c ~(l)~> c'.
\end{lstlisting}


\subsection{Definition of Fairness}
Then we can define \texttt{fairness} predicate for transition path.
For the given transition path and for each label, if \texttt{infinitely often enabled} holds, then \texttt{eventually processed} holds.
\texttt{is postfix of} predicate is used for representing 'infinite'.
If \texttt{is postfix of} is not used, the transition may not be processed after the transition is processed although the transition is processed in whole the path.
To prevent it, if \texttt{inifinitely often enabled} holds then \texttt{eventually processed} holds for arbitrary postfix path by using \texttt{is postfix path}.

\begin{lstlisting}
Definition is_postfix_of
    (p' p : path) : Prop :=
  exists n, (forall m, p' m = p (m + n)).

Definition fairness : Prop :=
  forall p p', is_postfix_of p' p ->
    (forall l,
      infinitely_often_enabled l p' ->
      eventually_processed l p').
\end{lstlisting}

\section{Extraction}
\label{sec:extraction}

Extraction is a Coq feature which enables to convert Coq programs to the programs of other languages.
Normal Coq can extract programs to OCaml, Haskell, and Scheme.
If we want to extract to other languages or use custom extraction algorithm, we have to implement it as plugins or patches.
Actario has custom extraction mechanism for the programs using Actario.
It can extract to Erlang.
It is not proven that the extraction mechanism does not change the meanings of Actario programs and Erlang programs.
In Actario, \lstinline|ActorExtraction| command is defined for extracting actor systems.
It is used like traditional \lstinline|Extraction| command.

\subsection{Data Types}

Values of algebraic data types are extracted to a tuple with the label.
Value constructor is extracted to a label, and arguments are extracted to the second and the following elements of the tuple.
Figure \ref{coq:adt} is an example of extraction of the natural number type.

\begin{figure}[t]
  \begin{lstlisting}
    (* Inductive nat :=  *)
    (* | O : nat         *)
    (* | S : nat -> nat. *)

    O (* => {o} *)
    S (S (S O)) (* => {s, {s, {s, o}}} *)
  \end{lstlisting}
  \caption{example of extraction of algebraic data types}\label{coq:adt}
\end{figure}

However, actions of actors, for example, \textsf{send}, \textsf{new}, \textsf{self}, \textsf{become} and \texttt{behavior} are implemented as value constructor of \texttt{actions} and \texttt{behavior} type
We handle these constructors as special to generate the corresponding syntax of Erlang.

%% ただし、アクターモデル特有である動作、例えば send, new, self, become, receive は、Actario では\texttt{actions}, \texttt{behavior} 型の値コンストラクタとして実装されているので、これだとただのタプルのデータとして抽出されてしまう。
%% このようなアクションは Erlang では構文または関数となっているため、このようなアクションだけを特別扱いして、対応する Erlang のものに変換する必要がある。

For example, Actario code shown in figure \ref{coq:extractionex} is extracted to Erlang code shown in figure \ref{erl:extractionex}.

\begin{figure}[t]
  \begin{lstlisting}
CoFixpoint behvA :=
  receive (fun msg =>
    match msg with
      | name_msg sender =>
        me <- self;
        sender ! name_msg me;
        become behvA
      | _ =>
        child <- new behvB;
        child ! msg;
        become behvA
    end)
  \end{lstlisting}
  \caption{Extraction example: Actario code}\label{coq:extractionex}
\end{figure}

\begin{figure}[t]
  \begin{lstlisting}[language=erlang]
behvA() ->
  receive Msg -> case Msg of
    {name_msg, Sender} ->
      Me = self(),
      Sender ! {name_msg, Me},
      behvA()
    _ ->
      Child = spawn(fun() -> behvB() end),
      Child ! Msg
      behvA()
  end.
  \end{lstlisting}
  \caption{Extraction example: Erlang code}\label{erl:extractionex}
\end{figure}

\subsection{Name}
In Actario, a programmer does not make actor names from constructors, so that all of actor names are in variables.
Therefore, all of actor names in extracted code are variables.
These variables are bound by values of name type in Actario, but in Erlang, these variables are bound by process ids.
%% 名前はプログラマが自分でコンストラクタから作るということはしないので、すべてが変数に格納されている。
%% なので Erlang に変換すると変数になっている。
%% Actario で書いたプログラムの name 型の変数には Actario での name 型の値に束縛されているが、Erlang に変換したあとのプログラムの対応する変数は Erlang のプロセスIDに束縛されている。

\subsection{Execution}
The program extracted by Actario is impossible to execute by itself.
So Actario's programmers have to write executor to execute the extracted Actor system in Erlang.
For example, we consider factorial system described in section \ref{sec:overview}.

\begin{lstlisting}
Definition factorial_system (n : nat) (parent : name) : config :=
  init "factorial" (
         x <- new factorial_behv;
         x ! tuple_msg (nat_msg n) (name_msg parent);
         become empty_behv
       ).
\end{lstlisting}

\lstinline|factorial_system| is extracted to the following Erlang code.

\begin{lstlisting}[language=erlang]
factorial_system(N, Parent) ->
    X = spawn(fun() ->
        factorial_behv()
    end),
    X ! {tuple_msg, {nat_msg, N}, {name_msg, Parent}},
    empty_behv().
\end{lstlisting}

To execute this, we have to write executor like the following.
\lstinline[language=erlang]|nat2int| and \lstinline[language=erlang]|int2nat| are auxiliary functions for converting Coq's natural number and Erlang's integer.

\begin{lstlisting}[language=erlang]
-module(fact_main).
-export([fact/1]).

fact(N) ->
    _ = spawn(factorial, factorial_system, [int2nat(N), self()]),
    receive
        {nat_msg, Result} ->
            io:fwrite("fact(~w) = ~w~n", [N, nat2int(Result)]);
        _ ->
            io:fwrite("error~n")
    end.

nat2int({o}) -> 0;
nat2int({s, N}) -> nat2int(N) + 1.

int2nat(0) -> {o};
int2nat(N) when N > 0 -> {s, int2nat(N - 1)};
int2nat(_) -> {o}.
\end{lstlisting}

\section{Related Work}
\label{sec:relatedwork}

Appl\(\pi\) is a Coq library for modeling and verifying concurrent programs \cite{Affeldt200817}.
Actario is very inspired by Appl\(\pi\), for example, the definition of fairness, continuation passing style in \texttt{actions} and framework design.
The main difference of Appl\(\pi\) and Actario is that Appl\(\pi\) adopts \(\pi\)-calculus for its concurrent computation basic, but Actario adopts the Actor model for its concurrent computation basic.

Musser and Varela are formalized the Actor model in the Athena theorem prover \cite{Athena}\cite{Musser:2013aa}. % In this paper, transition path is defined as sequence of labels.
In this paper, name uniqueness is proven.
However, a programmer has to name new actors explicitly.
Therefore, a programmer has to select a fresh name. It is difficult to give always fresh name in complex system.
In addition, it is impossible to run the program built in the formalization, while Actario can by extraction.
%% これは transition path を遷移のラベルの列としている。
%% この形式化では遷移の間で名前の一意性が証明されているが、 % note: creating, trans-create, unique-ids-persistence in transition.ath
%% この形式化を用いてプログラミングする際には名前をプログラマが明示的に与えなければならないので、アクターを生成するときには名前が重複しないように注意深く設定しなければならない。
%% また、Actario の方がより実際のプログラミングを行うに近いプログラミングができる。
%% Extraction はない。

Verdi is a framework for constructing and verifying fault-tolerant distributed systems \cite{Verdi}.
A system assumed no network failure is converted to the system which tolerates dropping packets, duplication of packets, and machine failure.
One of the purposes of Actario is also to build and verify fault-tolerant distributed systems.
We will introduce \textit{supervisor} mechanism to achieve building fault-tolerant systems generally used in Erlang and Akka.
%% \note{Supervisor についての説明はいるかどうか}
%% Supervisor is used for fault-tolerance and rapid recovery in the system, introduced in Erlang, Akka, and so on.

%% operate correctly; preserving the properties of the system.
%% 故障がまったくない意味論上で作ったシステムを、メッセージのドロップや重複、マシンの以上終了などを含む意味論上でも正常に動作 (ここでいう正常に動作とは、そのシステムについて成り立っていてほしい性質が成り立ついるということ) するシステムへと変換する仕組みが備わっている。
%% Actario の目標の一つも分散システムの検証で、Actario では Erlang や Akka で採用されているような Supervisor を使った耐障害性のあるシステムに対しての検証を目指している。
%% Supervisor は Erlang や Akka で取り入れられている考え方で、

Tony Garnock-Jones, Sam Tobin-Hochstadt, and Matthias Felleisen give a formalization of the Actor model using Coq \cite{Garnock-Jones:2014aa}.
In this paper, the operational semantics is formalized so that transition is decidable.
Due to this, it is difficult to apply the formalization to realistic concurrent systems.

\section{Concluding Remarks}
\label{sec:conclusion}

\note{Overviewの最初にこのような記述があったが,ここかIntroductionあたりの方がよいかも}
Actario is currently under development and still does not provide
convenient libraries of predicates, lemmas, tactics and so
forth. Thus, verifying a user-defined actor system may involve a large
amount of work. However, we have formally proved that the underlying
execution model provided in the framework satisfies three important
actor properties: name uniqueness, actor persistence and name
persistence. The fact implies that a system described using Actario is
guaranteed to have these actor properties.


\appendix
\section{Labeled Transition Semantics in Actario}\label{app:lts}
\begin{figure}[b]
\begin{lstlisting}[style=small]
Reserved Notation "c1 '~(' t ')~>' c2" (at level 60).
Inductive trans : label -> config -> config -> Prop :=
(* receive transition *)
| trans_receive :
    forall to from content f gen sendings_l sendings_r actors_l actors_r,
      (sendings_l ++ Build_sending to from content :: sendings_r)
                 >< (actors_l ++ Build_actor to (become (receive f)) gen :: actors_r)
        ~(Receive to from content)~>
        (sendings_l ++ sendings_r) >< (actors_l ++ Build_actor to (f content) gen :: actors_r)
(* send transition *)
| trans_send :
    forall from to content cont gen sendings_l sendings_r actors_l actors_r,
      (sendings_l ++ sendings_r)
                 >< (actors_l ++ Build_actor from (send to content cont) gen :: actors_r)
         ~(Send from to content)~>
         (sendings_l ++ Build_sending to from content :: sendings_r)
           >< (actors_l ++ Build_actor from cont gen :: actors_r)
(* new transition *)
| trans_new :
    forall parent behv cont gen sendings actors_l actors_r,
      sendings >< (actors_l ++ Build_actor parent (new behv cont) gen :: actors_r)
        ~(New (generated gen parent))~>
        sendings ><
          (Build_actor (generated gen parent) (become behv) 0 ::
          actors_l ++
          Build_actor parent (cont (generated gen parent)) (S gen) ::
          actors_r)
(* self transition *)
| trans_self :
    forall me cont gen sendings actors_l actors_r,
      sendings >< (actors_l ++ Build_actor me (self cont) gen :: actors_r)
        ~(Self me)~>
        sendings >< (actors_l ++ Build_actor me (cont me) gen :: actors_r)
where "c1 '~(' t ')~>' c2" := (trans t c1 c2).
\end{lstlisting}
\caption{Labeled Transition Semantics in Actario}\label{fig:coq:semantics}
\end{figure}

% We recommend abbrvnat bibliography style.

\bibliographystyle{abbrvnat}
\bibliography{sigproc}

% The bibliography should be embedded for final submission.

%% \begin{thebibliography}{}
%% \softraggedright

%% \bibitem[Smith et~al.(2009)Smith, Jones]{smith02}
%% P. Q. Smith, and X. Y. Jones. ...reference text...

%% \end{thebibliography}


\end{document}

%                       Revision History
%                       -------- -------
%  Date         Person  Ver.    Change
%  ----         ------  ----    ------

%  2013.06.29   TU      0.1--4  comments on permission/copyright notices
