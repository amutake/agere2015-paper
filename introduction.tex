\section{Introduction}
\label{sec:introduction}


The Actor model\cite{Agha:1986aa} is a kind of concurrent computation
model, in which a system is expressed as a collection of autonomous
computing entities called actors that communicate each other only with
asynchronous messages.
On receiving a message, an actor may (1)
send messages to other actors (or itself) whose names are known to the
sender, (2) create new actors and (3) change its behavior for the next
message.

Starting from the 1970s, the Actor model and its variations such as
concurrent objects\cite{Yonezawa:1986aa} have a long research
history. They are today regarded as popular high-level abstractions
for concurrent and parallel programming used in some industrial
strength language and libraries such as Erlang\cite{Erlang},
Scala\cite{Scala} and Akka\cite{Akka}.  Because of this situation,
establishing a mechanized formal verification method for actor-based
systems is a pressing issue.

Several methods and systems for formally verifying actor-based systems
have been presented recently. Rebeca\cite{Sirjani:2011aa} is a
modeling language that allows model-checking.  For deductive
verification using proof assistants, formalizations using
Athena\cite{Musser:2013aa} and Coq\cite{Garnock-Jones:2014aa} have
been presented.

A \emph{name}\footnote{The term \emph{mail address} is used in some other
  literature.} in the Actor model is a unique conceptual location
associated with each actor.  The concept of \emph{name uniqueness}
denotes that each name uniquely refers an actor and each actor should
be referred by a single name.  In the implementations of actor systems
including Erlang, Scala, and Akka, naming of actors is implicit; we
don't need to manually assign a fresh name to a newly created actor.
The name uniquness may be broken if the naming is explicit in complex
systems.  Implicit naming, however, might complicates the formal
treatment of actor-based systems. Thus, some formalization adopts
explicit naming.


In this paper, we propose Actario\cite{Actario}, a Coq framework for
implementing and verifying actor-based systems.  The framework (1)
supports Erlang-like notation for describing an actor system, (2)
allows verifying desired properties of the system using the proof
mechanism in Coq, and (3) generates executable Erlang code from the
system description.

To be close to realistic actor languages and libraries, we designed
Actario to support implicit naming. The naming mechanism behind the
scene is formally proven to satisfy the name uniqueness.  We also
proved other properties including the persistence of actors and
messages.  The proof scripts of these properties are available in the
GitHub repository of Actario\cite{Actario}.


% 形式的検証をおこなう上での問題
% 特にアクターの名前の生成,一貫性

%% 特にアクターの名前の
%% アクターの名前は必ず一意である必要がある。
%% アクターはシステムの進行に伴って動的に生成されるが、その中で fresh でない名前の生成が行われることはない。
%% fresh なアクターの名前の付け方には、~がある。

% ここで解決したい問題は何か(問題)
% * アクターモデルは並行計算のモデルとして real world でよく使われている
% * 一般的に並行システムの検証は難しい
% * アクターシステム (アクターモデルで記述されたアプリケーション) の検証を行いたい
% Actarioではどのようにして問題を解決しようとしているか(手法)
% 提案手法
% * 名前付けの方法とそれによる名前の一貫性の形式的証明
% それによってどのようなことが明らかになるか:
% * 普通のプログラミングと同様にモデル化できる
% * 名前を勝手につけられるので検証は難しくなる(?)

% Actarioの特徴
% * 普通のプログラミングと同様にシステムを記述できる
% * Coqによって検証もできる
%   + 実装しているシステム自身はCoqによって検証されている
%     - 名前の一貫性(済)
%     - アクターのpersistence (済)
%     - メッセージのpersistence (済)
% * Erlangプログラムを生成できる

The layout of the rest of this paper is as follows.
The next section describes the overview of Actario.
In Section~\ref{sec:semantics}, we give the operational semantics of the Actor model formalized in Actario.
Section~\ref{sec:uniqueness} outlines the proof of the uniqueness property on dynamically generated names.
In Section~\ref{sec:fairness}, we discuss fairness properties formalized in Actario.
The code extraction mechanism is described in Section~\ref{sec:extraction}.
Finally, Section~\ref{sec:relatedwork} overviews related work and Section~\ref{sec:conclusion} concludes the paper.

%% 2節では Actario で使うアクターのシンタックスを説明する。
%% 3節では Actario で用いるアクターモデルの意味論について説明する。
%% 4節では、Actario が用いる意味論において、動的に生成されるアクターの名前が一意になるという定理の証明を説明する。
%% 5節では Actario における fairness の定式化について説明し、
%% 6節で他のアクターモデルの形式化や並行・分散システムの検証用フレームワークとの差異を述べる。
