\section{Introduction}
アクターモデルは並行計算のモデルの一つである。
アクターモデルではアクターと呼ばれる計算実体が非同期にメッセージを送り合うことで計算を進める。
あるアクターがメッセージに対してどういう動作を行うか、ということをそのアクターの振る舞いという。
アクターはメッセージを受け取ると、新しいアクターを作る、他のアクターにメッセージを送る、自分自身の振る舞いを変える、という3つの動作を行うことができる。

アクターの名前は必ず一意である必要がある。
アクターはシステムの進行に伴って動的に生成されるが、その中で fresh でない名前の生成が行われることはない。
fresh なアクターの名前の付け方には、~がある。

%% jssst2014のintroから
アクターモデルに基づく言語は多数提案・実装されてきており、それらの研究成果をもとに現在 Erlang、Akka 等、アクターモデルを並行計算の基盤としたプログラミング言語やライブラリが実用に供されている。
そのため、アクターモデルに寄って構成されたシステムの形式的検証は喫緊の課題であると考えられる。

形式的検証については、アクターで記述されたシステムのモデル検査を可能にするモデル記述言語 Rebeca、
証明支援系 Athena を用いた形式化、
および Coq を用いたアクターモデルの定式化など、
最近になっていくつかの研究成果が出ている。
本研究では、アクターモデルの

Athena を用いた形式化では、
