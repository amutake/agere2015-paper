\section{semantics}
\label{sec:semantics}

本節では Actario が定義しているアクターモデルの操作的意味論について説明する。
まず意味論の説明に必要な \texttt{name} 型、\texttt{actor} 型、\texttt{in\_flight\_message} 型、\texttt{config} 型について説明し、
操作的意味について説明する。

\subsection{Actor Name}
Actario ではアクターの名前は、親がいないアクターと何らかのアクターから生成されたアクターに分けて定義する (図\ref{coq:name})。
親がいないアクターは toplevel actor と呼ぶ。
これはアクターシステムに最初から存在するアクターを表す。
何らかのアクターから生成されたアクターを generated actor と呼ぶ。
generated actor の名前は、親アクターの名前と、その親アクターが何番目に生成した子かという番号 (generation number) のペアとする。
これは動的に生成されうるアクターの名前の一意性を保つためである。

\begin{figure}[tb]
  \begin{lstlisting}
    Inductive name : Set :=
    | toplevel : string -> name
    | generated : nat -> name -> name.
  \end{lstlisting}
  \caption{name}\label{coq:name}
\end{figure}


\subsection{actor}
アクターは、自分自身名前、残りのアクションの列、次回アクターを生成する際に使う generation number の3つのレコード型である (図\ref{coq:actor})。
残りのアクションの列が become のみの場合、このアクターはメッセージを受け取れる状態にあるということを表す。

\begin{figure}[tb]
  \begin{lstlisting}
    Record actor := {
      actor_name : name;
      remaining_actions : actions;
      next_num : gen_number
    }.
  \end{lstlisting}
  \caption{actor}\label{coq:actor}
\end{figure}

\subsection{in flight message}
宛先と送り主とメッセージの内容からなるレコード型である (図\ref{coq:inflight}))。
まだ受け取られていないメッセージを表す。
configuration に用いる。

\begin{figure}[tb]
  \begin{lstlisting}
    Record in_flight_message := {
      to : name;
      from : name;
      content : message
    }.
  \end{lstlisting}
  \caption{in flight message}\label{coq:inflight}
\end{figure}

\subsection{configuration}
configuration はアクターシステムの現在のスナップショットを表す。
configuration はアクターモデルの意味論を定義する際に用いるものである。
Actario では configuration を actor の列と in\_flight\_message の列のペアとして定義する
\footnote{seq は ssreflect で定義されているもので、Coq の list の notation。}。

\begin{figure}[tb]
  \begin{lstlisting}
    Record config := {
      in_flight_messages : seq in_flight_message;
      actors : seq actor
    }.
  \end{lstlisting}
  \caption{config}\label{coq:config}
\end{figure}


\subsection{label}

Actario ではアクターモデルの操作的意味を labeled transition system として定式化するため、ラベルを定義する (図\ref{coq:label}))。
以下にそれぞれの説明を示す。

\begin{description}[style=nextline,leftmargin=12pt,parsep=0pt]
\item[\texttt{Receive (to : name) (from : name) (content : message)}]
  \texttt{to} という名前のアクターが \texttt{from} という名前のアクターからのメッセージ \texttt{content} を受け取って遷移したということを表す。
\item[\texttt{Send (from : name) (to : name) (content : message)}]
  \texttt{from} という名前のアクターが \texttt{to} という名前のアクターに向けてメッセージ \texttt{content} を送ってシステムが遷移したということを表す。
\item[\texttt{New (child : name)}]
  \texttt{child} という名前の新しいアクターが生成されてシステムが遷移したということを表す。
\item[\texttt{Self (me : name)}]
  \texttt{me} という名前のアクターが自分自身の名前を読みだしたということを表す。
\end{description}


\begin{figure}[tb]
  \begin{lstlisting}
    Inductive label :=
    | Receive (to : name) (from : name) (content : message) (* `to` receives a message `content` from `from` *)
    | Send (from : name) (to : name) (content : message)    (* `from` sends a message `content` to `to` *)
    | New (child : name)                                    (* an actor named `child` is created *)
    | Self (me : name).                                     (* `me` gets own name *)
  \end{lstlisting}
  \caption{label}\label{coq:label}
\end{figure}


\subsection{semantics}

アクターモデルの操作的意味を configuration のラベル付き遷移システムとして定式化する。
これ以降用いる記号を図\ref{fig:config}のように定義する。

\begin{figure}[tb]
  \begin{displaymath}
    \begin{array}{rclcl}
      c & \in & \textit{Configuration} & =   & \textit{InFlight} \times \textit{Actor} \\
      a & \in & \textit{Actor}  & =   & \textit{Name} \times \textit{Actions} \times \mathbb{N} \\
      n & \in & \textit{Name}   & ::= & \textsf{toplevel}(s) \mid \textsf{generated}(g, n) \\
      m & \in & \textit{Message} & =  & \textit{Name} + \textit{PrimVal} + \\
        &     &                 &     & \textit{Message} \times \cdots \times \textit{Message} \\
      i & \in & \textit{InFlight} & = & \textit{Name} \times \textit{Name} \times \textit{Message} \\
      b & \in & \textit{Behavior} & = & \textit{Message} \rightarrow \textit{Actions} \\
      \alpha & \in & \textit{Actions} & ::= & \textsf{send}(n, m, \alpha) \\
        &     &                 &   | & \textsf{new}(b, \kappa) \\
        &     &                 &   | & \textsf{self}(\kappa) \\
        &     &                 &   | & \textsf{become}(b) \\
      l & \in & \textit{Label}  & ::= & \textsf{Receive}(n, n, m) \\
        &     &                 &   | & \textsf{Send}(n, n, m) \\
        &     &                 &   | & \textsf{New}(n) \\
        &     &                 &   | & \textsf{Self}(n) \\
      \kappa & \in & \textit{Name} \rightarrow \textit{Actions} & & \\
      g & \in & \mathbb{N} & &
    \end{array}
  \end{displaymath}
  \caption{Configuration}\label{fig:config}
\end{figure}


このラベル一つ一つに対応した意味論は図\ref{semantics}にある。
receive はメッセージ待ち状態にあるアクターが、自身に向けて送られたメッセージを受け取り、アクションの列を生成する遷移である。
send はあるアクターが他のアクターに向けてメッセージを送る遷移。
...
Actario での定義は Appendix \ref{app:lts} にある。

\begin{figure*}[t]
  \begin{displaymath}
    \begin{array}{rcll}
      (I \uplus \{(n_{\textrm{to}}, n_{\textrm{from}}, m)\}, A \cup \{(n_{\textrm{to}}, \textsf{become}(b), g)\}) &
      \overset{\textsf{Receive}(n_{\textrm{to}}, n_{\textrm{from}}, m)}{\leadsto} &
      (I, A \cup \{(n_{\textrm{to}}, b(m), g)\}) &
      \textsc{(Receive)} \\[1ex]

      (I, A \cup \{(n_{\textrm{from}}, \textsf{send}(n_{\textrm{to}}, m, \alpha), g)\}) &
      \overset{\textsf{Send}(n_{\textrm{from}}, n_{\textrm{to}}, m)}{\leadsto} &
      (I \uplus \{(n_{\textrm{to}}, n_{\textrm{from}}, m)\}, A \cup \{(n_{\textrm{from}}, \alpha, g)\}) &
      \textsc{(Send)} \\[1ex]

      (I, A \cup \{(n, \textsf{new}(b, \kappa), g)\}) &
      \overset{\textsf{New}(n')}{\leadsto} &
      (I, A \cup \{(n, \kappa(n'), g + 1), (n', \textsf{become}(b), 0)\}) & \\
      & & \hfill \textrm{where}\ n' := \textsf{generated}(g, n) &
      \textsc{(New)} \\[1ex]

      (I, A \cup \{(n, \textsf{self}(\kappa), g)\}) &
      \overset{\textsf{Self}(n)}{\leadsto} &
      (I, A \cup \{n, \kappa(n), g\}) &
      \textsc{(Self)}
    \end{array}
  \end{displaymath}
  \caption{labeled transition semantics}\label{fig:semantics}
\end{figure*}
