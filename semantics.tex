\section{semantics}

操作的意味論を labeled transition system として形式化する。
ラベルとして receive, send, new, self の4つを用意する。

\begin{lstlisting}
  Inductive label :=
  | Receive (to : name) (from : name) (content : message) (* `to` receives a message `content` from `from` *)
  | Send (from : name) (to : name) (content : message)    (* `from` sends a message `content` to `to` *)
  | New (child : name)                                    (* an actor named `child` is created *)
  | Self (me : name).                                     (* `me` gets own name *)
\end{lstlisting}

\texttt{Receive (to : name) (from : name) (content : message)} は to という名前のアクターが from という名前のアクターからのメッセージ content を受け取ったということを表す。
\texttt{Send (from : name) (to : name) (content : message)} は from という名前のアクターが to という名前のアクターに向けてメッセージ content を送ったということを表す。
\texttt{New (child : name)} は child という名前の新しいアクターを生成したということを表し、
\texttt{Self (me : name)} は me という名前のアクターが自分自身の名前を読みだしたということを表す。

このラベル一つ一つに対応した意味論は図\ref{semantics}にある。
receive はメッセージ待ち状態にあるアクターが、自身に向けて送られたメッセージを受け取り、アクションの列を生成する遷移である。
send はあるアクターが他のアクターに向けてメッセージを送る遷移。
...
これを Coq 上で定義すると図\ref{coq:semantics}のようになる。ほとんどそのままだが集合の代わりにリストを使っている。なぜならば...

\begin{figure}[h]
  \infrule[Receive]{
  }{
    sendings \uplus \{Sending(to, from, content)\} \bowtie actors \cup \{Actor(to, become(behv), next)\} \leadsto^{Receive(to, from, content)}
    sendings \bowtie actors \cup \{Actor(to, behv(content), next)\}
  }
  \infrule[Send]{
  }{
    sendings \bowtie actors \cup \{Actor(from, to \ ! \ content; cont, next)\} \leadsto^{Send(from, to, content)}
    sendings \uplus \{Sending(to, from, content)\} \bowtie actors \cup \{Actor(from, cont, next)\}
  }
  \infrule[New]{
  }{
    sendings \bowtie actors \cup \{Actor(parent, n \leftarrow new behv; cont, next)\} \leadsto^{New(child)}
    sendings \bowtie actors \cup \{Actor(parent, cont{n/child}, next + 1), Actor(child, become(behv), 0)\}
    where child := Generated(next, parent)
  }
  \infrule[Self]{
  }{
    sendings \bowtie actors \cup \{Actor(me, n \leftarrow self; cont, next)\} \leadsto^{Self(me)}
    sendings \bowtie actors \cup \{Actor(me, cont{n/me}, next)\}
  }
\end{figure}




\begin{lstlisting}
Reserved Notation "c1 '~(' t ')~>' c2" (at level 60).
Inductive trans : label -> config -> config -> Prop :=
(* receive transition *)
| trans_receive :
    forall to from content f gen sendings_l sendings_r actors_l actors_r,
      (sendings_l ++ Build_sending to from content :: sendings_r)
                 >< (actors_l ++ Build_actor to (become (receive f)) gen :: actors_r)
        ~(Receive to from content)~>
        (sendings_l ++ sendings_r) >< (actors_l ++ Build_actor to (f content) gen :: actors_r)
(* send transition *)
| trans_send :
    forall from to content cont gen sendings_l sendings_r actors_l actors_r,
      (sendings_l ++ sendings_r)
                 >< (actors_l ++ Build_actor from (send to content cont) gen :: actors_r)
         ~(Send from to content)~>
         (sendings_l ++ Build_sending to from content :: sendings_r)
           >< (actors_l ++ Build_actor from cont gen :: actors_r)
(* new transition *)
| trans_new :
    forall parent behv cont gen sendings actors_l actors_r,
      sendings >< (actors_l ++ Build_actor parent (new behv cont) gen :: actors_r)
        ~(New (generated gen parent))~>
        sendings ><
          (Build_actor (generated gen parent) (become behv) 0 ::
          actors_l ++
          Build_actor parent (cont (generated gen parent)) (S gen) ::
          actors_r)
(* self transition *)
| trans_self :
    forall me cont gen sendings actors_l actors_r,
      sendings >< (actors_l ++ Build_actor me (self cont) gen :: actors_r)
        ~(Self me)~>
        sendings >< (actors_l ++ Build_actor me (cont me) gen :: actors_r)
where "c1 '~(' t ')~>' c2" := (trans t c1 c2).
\end{lstlisting}
