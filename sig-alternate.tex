% This is "sig-alternate.tex" V2.0 May 2012
% This file should be compiled with V2.5 of "sig-alternate.cls" May 2012
%
% This example file demonstrates the use of the 'sig-alternate.cls'
% V2.5 LaTeX2e document class file. It is for those submitting
% articles to ACM Conference Proceedings WHO DO NOT WISH TO
% STRICTLY ADHERE TO THE SIGS (PUBS-BOARD-ENDORSED) STYLE.
% The 'sig-alternate.cls' file will produce a similar-looking,
% albeit, 'tighter' paper resulting in, invariably, fewer pages.
%
% ----------------------------------------------------------------------------------------------------------------
% This .tex file (and associated .cls V2.5) produces:
%       1) The Permission Statement
%       2) The Conference (location) Info information
%       3) The Copyright Line with ACM data
%       4) NO page numbers
%
% as against the acm_proc_article-sp.cls file which
% DOES NOT produce 1) thru' 3) above.
%
% Using 'sig-alternate.cls' you have control, however, from within
% the source .tex file, over both the CopyrightYear
% (defaulted to 200X) and the ACM Copyright Data
% (defaulted to X-XXXXX-XX-X/XX/XX).
% e.g.
% \CopyrightYear{2007} will cause 2007 to appear in the copyright line.
% \crdata{0-12345-67-8/90/12} will cause 0-12345-67-8/90/12 to appear in the copyright line.
%
% ---------------------------------------------------------------------------------------------------------------
% This .tex source is an example which *does* use
% the .bib file (from which the .bbl file % is produced).
% REMEMBER HOWEVER: After having produced the .bbl file,
% and prior to final submission, you *NEED* to 'insert'
% your .bbl file into your source .tex file so as to provide
% ONE 'self-contained' source file.
%
% ================= IF YOU HAVE QUESTIONS =======================
% Questions regarding the SIGS styles, SIGS policies and
% procedures, Conferences etc. should be sent to
% Adrienne Griscti (griscti@acm.org)
%
% Technical questions _only_ to
% Gerald Murray (murray@hq.acm.org)
% ===============================================================
%
% For tracking purposes - this is V2.0 - May 2012

\documentclass{sig-alternate}

\begin{document}
%
% --- Author Metadata here ---
\conferenceinfo{WOODSTOCK}{'97 El Paso, Texas USA}
%\CopyrightYear{2007} % Allows default copyright year (20XX) to be over-ridden - IF NEED BE.
%\crdata{0-12345-67-8/90/01}  % Allows default copyright data (0-89791-88-6/97/05) to be over-ridden - IF NEED BE.
% --- End of Author Metadata ---

\title{Alternate {\ttlit ACM} SIG Proceedings Paper in LaTeX
Format\titlenote{(Produces the permission block, and
copyright information). For use with
SIG-ALTERNATE.CLS. Supported by ACM.}}
\subtitle{[Extended Abstract]
\titlenote{A full version of this paper is available as
\textit{Author's Guide to Preparing ACM SIG Proceedings Using
\LaTeX$2_\epsilon$\ and BibTeX} at
\texttt{www.acm.org/eaddress.htm}}}
%
% You need the command \numberofauthors to handle the 'placement
% and alignment' of the authors beneath the title.
%
% For aesthetic reasons, we recommend 'three authors at a time'
% i.e. three 'name/affiliation blocks' be placed beneath the title.
%
% NOTE: You are NOT restricted in how many 'rows' of
% "name/affiliations" may appear. We just ask that you restrict
% the number of 'columns' to three.
%
% Because of the available 'opening page real-estate'
% we ask you to refrain from putting more than six authors
% (two rows with three columns) beneath the article title.
% More than six makes the first-page appear very cluttered indeed.
%
% Use the \alignauthor commands to handle the names
% and affiliations for an 'aesthetic maximum' of six authors.
% Add names, affiliations, addresses for
% the seventh etc. author(s) as the argument for the
% \additionalauthors command.
% These 'additional authors' will be output/set for you
% without further effort on your part as the last section in
% the body of your article BEFORE References or any Appendices.

\numberofauthors{8} %  in this sample file, there are a *total*
% of EIGHT authors. SIX appear on the 'first-page' (for formatting
% reasons) and the remaining two appear in the \additionalauthors section.
%
\author{
% You can go ahead and credit any number of authors here,
% e.g. one 'row of three' or two rows (consisting of one row of three
% and a second row of one, two or three).
%
% The command \alignauthor (no curly braces needed) should
% precede each author name, affiliation/snail-mail address and
% e-mail address. Additionally, tag each line of
% affiliation/address with \affaddr, and tag the
% e-mail address with \email.
%
% 1st. author
\alignauthor
Ben Trovato\titlenote{Dr.~Trovato insisted his name be first.}\\
       \affaddr{Institute for Clarity in Documentation}\\
       \affaddr{1932 Wallamaloo Lane}\\
       \affaddr{Wallamaloo, New Zealand}\\
       \email{trovato@corporation.com}
% 2nd. author
\alignauthor
G.K.M. Tobin\titlenote{The secretary disavows
any knowledge of this author's actions.}\\
       \affaddr{Institute for Clarity in Documentation}\\
       \affaddr{P.O. Box 1212}\\
       \affaddr{Dublin, Ohio 43017-6221}\\
       \email{webmaster@marysville-ohio.com}
% 3rd. author
\alignauthor Lars Th{\o}rv{\"a}ld\titlenote{This author is the
one who did all the really hard work.}\\
       \affaddr{The Th{\o}rv{\"a}ld Group}\\
       \affaddr{1 Th{\o}rv{\"a}ld Circle}\\
       \affaddr{Hekla, Iceland}\\
       \email{larst@affiliation.org}
\and  % use '\and' if you need 'another row' of author names
% 4th. author
\alignauthor Lawrence P. Leipuner\\
       \affaddr{Brookhaven Laboratories}\\
       \affaddr{Brookhaven National Lab}\\
       \affaddr{P.O. Box 5000}\\
       \email{lleipuner@researchlabs.org}
% 5th. author
\alignauthor Sean Fogarty\\
       \affaddr{NASA Ames Research Center}\\
       \affaddr{Moffett Field}\\
       \affaddr{California 94035}\\
       \email{fogartys@amesres.org}
% 6th. author
\alignauthor Charles Palmer\\
       \affaddr{Palmer Research Laboratories}\\
       \affaddr{8600 Datapoint Drive}\\
       \affaddr{San Antonio, Texas 78229}\\
       \email{cpalmer@prl.com}
}
% There's nothing stopping you putting the seventh, eighth, etc.
% author on the opening page (as the 'third row') but we ask,
% for aesthetic reasons that you place these 'additional authors'
% in the \additional authors block, viz.
\additionalauthors{Additional authors: John Smith (The Th{\o}rv{\"a}ld Group,
email: {\texttt{jsmith@affiliation.org}}) and Julius P.~Kumquat
(The Kumquat Consortium, email: {\texttt{jpkumquat@consortium.net}}).}
\date{30 July 1999}
% Just remember to make sure that the TOTAL number of authors
% is the number that will appear on the first page PLUS the
% number that will appear in the \additionalauthors section.

\maketitle
\begin{abstract}
This paper provides a sample of a \LaTeX\ document which conforms,
somewhat loosely, to the formatting guidelines for
ACM SIG Proceedings. It is an {\em alternate} style which produces
a {\em tighter-looking} paper and was designed in response to
concerns expressed, by authors, over page-budgets.
It complements the document \textit{Author's (Alternate) Guide to
Preparing ACM SIG Proceedings Using \LaTeX$2_\epsilon$\ and Bib\TeX}.
This source file has been written with the intention of being
compiled under \LaTeX$2_\epsilon$\ and BibTeX.

The developers have tried to include every imaginable sort
of ``bells and whistles", such as a subtitle, footnotes on
title, subtitle and authors, as well as in the text, and
every optional component (e.g. Acknowledgments, Additional
Authors, Appendices), not to mention examples of
equations, theorems, tables and figures.

To make best use of this sample document, run it through \LaTeX\
and BibTeX, and compare this source code with the printed
output produced by the dvi file. A compiled PDF version
is available on the web page to help you with the
`look and feel'.
\end{abstract}

% A category with the (minimum) three required fields
\category{H.4}{Information Systems Applications}{Miscellaneous}
%A category including the fourth, optional field follows...
\category{D.2.8}{Software Engineering}{Metrics}[complexity measures, performance measures]

\terms{Theory}

\keywords{ACM proceedings, \LaTeX, text tagging}

\section{Introduction}
\label{sec:introduction}


The Actor model\cite{Agha:1986aa} is a kind of concurrent computation
model, in which a system is expressed as a collection of autonomous
computing entities called actors that communicate each other only with
asynchronous messages.
On receiving a message, an actor may (1)
send messages to other actors (or itself) whose names are known to the
sender, (2) create new actors and (3) change its behavior for the next
message.

Starting from the 1970s, the Actor model and its variations such as
concurrent objects\cite{Yonezawa:1986aa} have a long research
history. They are today regarded as popular high-level abstractions
for concurrent and parallel programming used in some industrial
strength language and libraries such as Erlang\cite{Erlang},
Scala\cite{Scala} and Akka\cite{Akka}.  Because of this situation,
establishing a mechanized formal verification method for actor-based
systems is a pressing issue.

Several methods and systems for formally verifying actor-based systems
have been presented recently. Rebeca\cite{Sirjani:2011aa} is a
modeling language that allows model-checking.  For deductive
verification using proof assistants, formalizations using
Athena\cite{Musser:2013aa} and Coq\cite{Garnock-Jones:2014aa} have
been presented.

A \emph{name}\footnote{The term \emph{mail address} is used in some other
  literature.} in the Actor model is a unique conceptual location
associated with each actor.  The concept of \emph{name uniqueness}
denotes that each name uniquely refers an actor and each actor should
be referred by a single name.  In the implementations of actor systems
including Erlang, Scala, and Akka, naming of actors is implicit; we
don't need to manually assign a fresh name to a newly created actor.
The name uniquness may be broken if the naming is explicit in complex
systems.  Implicit naming, however, might complicates the formal
treatment of actor-based systems. Thus, some formalization adopts
explicit naming.


In this paper, we propose Actario\cite{Actario}, a Coq framework for
implementing and verifying actor-based systems.  The framework (1)
supports Erlang-like notation for describing an actor system, (2)
allows verifying desired properties of the system using the proof
mechanism in Coq, and (3) generates executable Erlang code from the
system description.

To be close to realistic actor languages and libraries, we designed
Actario to support implicit naming. This is the main difference between
our formalization and formalizations using Athena\cite{Musser:2013aa}
or Coq\cite{Garnock-Jones:2014aa}. The naming mechanism behind the scene
is formally proved to satisfy the name uniqueness.  We also proved other
properties including the persistence of actors and messages.  The proof
scripts of these properties are available in the GitHub repository of
Actario\cite{Actario}.


% 形式的検証をおこなう上での問題
% 特にアクターの名前の生成,一貫性

%% 特にアクターの名前の
%% アクターの名前は必ず一意である必要がある。
%% アクターはシステムの進行に伴って動的に生成されるが、その中で fresh でない名前の生成が行われることはない。
%% fresh なアクターの名前の付け方には、~がある。

% ここで解決したい問題は何か(問題)
% * アクターモデルは並行計算のモデルとして real world でよく使われている
% * 一般的に並行システムの検証は難しい
% * アクターシステム (アクターモデルで記述されたアプリケーション) の検証を行いたい
% Actarioではどのようにして問題を解決しようとしているか(手法)
% 提案手法
% * 名前付けの方法とそれによる名前の一貫性の形式的証明
% それによってどのようなことが明らかになるか:
% * 普通のプログラミングと同様にモデル化できる
% * 名前を勝手につけられるので検証は難しくなる(?)

% Actarioの特徴
% * 普通のプログラミングと同様にシステムを記述できる
% * Coqによって検証もできる
%   + 実装しているシステム自身はCoqによって検証されている
%     - 名前の一貫性(済)
%     - アクターのpersistence (済)
%     - メッセージのpersistence (済)
% * Erlangプログラムを生成できる

The layout of the rest of this paper is as follows.
The next section describes the overview of Actario.
In Section~\ref{sec:semantics}, we give the operational semantics of the Actor model formalized in Actario.
Section~\ref{sec:uniqueness} outlines the proof of the uniqueness property on dynamically generated names.
In Section~\ref{sec:fairness}, we discuss fairness properties formalized in Actario.
The code extraction mechanism is described in Section~\ref{sec:extraction}.
Finally, Section~\ref{sec:relatedwork} overviews related work and Section~\ref{sec:conclusion} concludes the paper.

%% 2節では Actario で使うアクターのシンタックスを説明する。
%% 3節では Actario で用いるアクターモデルの意味論について説明する。
%% 4節では、Actario が用いる意味論において、動的に生成されるアクターの名前が一意になるという定理の証明を説明する。
%% 5節では Actario における fairness の定式化について説明し、
%% 6節で他のアクターモデルの形式化や並行・分散システムの検証用フレームワークとの差異を述べる。

\section{Syntax}

Actario では

以下のように定義する。

\begin{align*}
  configuration & ::= multiset of in\_flight\_message \bowtie set of actor \\
  actor & ::= Actor(name, actions, \mathbb{N}) \\
  name & ::= toplevel | generated(\mathbb{N}, name) \\
  actions & ::= name ! message; actions \\
  & \mid new behavior
\end{align*}


\subsection{Actor Name}
アクターの名前は、親がいないアクターと何らかのアクターから生成されたアクターに分けて定義する。
親がいないアクターは toplevel actor と呼ぶ。
これはアクターシステムに最初から存在するアクターを表す。
何らかのアクターから生成されたアクターを generated actor と呼ぶ。
generated actor の名前は、親アクターの名前と、その親アクターが何番目に生成した子かという番号 (generation number) のペアとする。
これは動的に生成されうるアクターの名前の一意性を保つためである。
基本的に Actario のプログラマは自分で名前を構築するようなことはしない。

\begin{lstlisting}
Inductive name : Set :=
| toplevel : string -> name
| generated : nat -> name -> name.
\end{lstlisting}

\subsection{message}
アクター間で取り交わされるメッセージの型を定義する。
Coq 内でメッセージを扱うため、Coq の型として定義する。
また、任意の型の値をメッセージとすることはできない。
これは、
1. これを導入すると name, 後述する actor, actions, behavior にメッセージの型パラメータを付与しなければいけなくなり、この後の証明が煩雑になる
2. ターゲットとして Erlang を想定しているので、Erlang のプリミティブをサポートするくらいでいい
3. この定義で十分
のためである。(理由が弱い)

\begin{lstlisting}
Inductive message : Set :=
| empty_msg : message
| name_msg : name -> message
| str_msg : string -> message
| nat_msg : nat -> message
| bool_msg : bool -> message
| tuple_msg : message -> message -> message.
\end{lstlisting}

\subsection{actions, behavior}
\texttt{actions} はアクターが行うアクションの列である。new, send, self, become がある。
behavior is a function from message to actions.

\begin{lstlisting}
CoInductive actions : Type :=
| new : behavior -> (name -> actions) -> actions
| send : name -> message -> actions -> actions
| self : (name -> actions) -> actions
| become : behavior -> actions
with behavior : Type :=
| receive : (message -> actions) -> behavior.
\end{lstlisting}

actions のそれぞれのコンストラクタを説明する。
\texttt{new : behavior -> (name -> actions) -> actions} は、新しいアクターを生成するアクションである。
引数として与えられたある behavior を initial behavior として新しいアクターを生成し、その名前をアクションの継続に渡す。
\texttt{send : name -> message -> actions -> actions} は、指定した名前のアクターにメッセージを送るアクションである。
第一引数に指定した名前のアクターに、第二引数で指定したメッセージを送り、第三引数で指定したその後のアクションを実行する、という意味である。
\texttt{self : (name -> actions) -> actions} は自分自身の名前を得るアクションである。
自分の名前を第一引数で与えられた継続に渡し、残りのアクションを得る。
\texttt{become : behavior -> actions} は次のメッセージに対して指定した振る舞いで処理する。
また、次のメッセージの待ち状態になることも表し、これ以降アクションは取れないようになっている。

actions と behavior は co-inductive types として定義される。
これは become で現在の behavior を指定したり、ある他の behavior と相互再帰するようなことがあるからである。
Coq ではこのような、コンストラクタの深さが小さくならない再帰は通常の inductive definition では書けなくなっている。

また、以下のように notation をつけている。

\begin{lstlisting}
Notation ``n '<-' 'new' behv ; cont'' := (new behv (fun n => cont)) (at level 0, cont at level 10).
Notation ``n '!' m ';' a'' := (send n m a) (at level 0, a at level 10).
Notation ``me '<-' 'self' ';' cont'' := (self (fun me => cont)) (at level 0, cont at level 10).
\end{lstlisting}

\texttt{n <- new behv; cont} は \texttt{new behv (fun n => cont)} に変換され、
\texttt{n ! msg; cont} は \texttt{send n msg cont} に変換され、
\texttt{me <- self; cont} は \texttt{self (fun me => cont)} に変換される。

\subsection{actor}
アクターは、自分自身名前、残りのアクションの列、次回アクターを生成する際に使う generation number の3つのレコード型である。
残りのアクションの列が become のみの場合、このアクターはメッセージを受け取れる状態にあるということを表す。

\begin{lstlisting}
Record actor := {
  actor_name : name;
  remaining_actions : actions;
  next_num : gen_number
}.
\end{lstlisting}

\subsection{in flight message}
宛先と送り主とメッセージの内容からなるレコード型である。
まだ受け取られていないメッセージを表す。
configuration に用いる。

\begin{lstlisting}
Record in_flight_message := {
  to : name;
  from : name;
  content : message
}.
\end{lstlisting}

\subsection{configuration}
configuration はアクターシステムの現在のスナップショットを表す。
configuration はアクターモデルの意味論を定義する際に用いるものである。
Actario では configuration を actor の列と in\_flight\_message の列のペアとして定義する。

\begin{lstlisting}
Record config := {
  in_flight_messages : seq in_flight_message;
  actors : seq actor
}.
\end{lstlisting}

\footnote{seq は ssreflect で定義されているもので、Coq の list の notation。}

\subsection{Example: factorial system}
例として階乗を計算するアクターシステムを定義する。
このシステムは、与えられた自然数 n に対して、n の階乗を計算した結果を、与えられた名前のアクターに対して送信するシステムである。
次に何をかけるかという情報を持った (continuation を持った) アクターを作るシステムになっている。\ref{agha:2004}

\begin{lstlisting}
CoFixpoint factorial_cont_behv (val : nat) (cust : name) : behavior :=
  receive (fun msg =>
         match msg with
           | nat_msg arg =>
             cust ! nat_msg (val * arg);
             become empty_behv
           | _ => become (factorial_cont_behv val cust)
         end).

CoFixpoint factorial_behv : behavior :=
  receive (fun msg =>
         match msg with
           | tuple_msg (nat_msg 0) (name_msg cust) =>
             cust ! nat_msg 1;
             become factorial_behv
           | tuple_msg (nat_msg (S n)) (name_msg cust) =>
             cont <- new (factorial_cont_behv (S n) cust);
             me <- self;
             me ! tuple_msg (nat_msg n) (name_msg cont);
             become factorial_behv
           | _ => become factorial_behv
         end).

Definition factorial_system (n : nat) (parent : name) : config :=
  init "factorial" (
         x <- new factorial_behv;
         x ! tuple_msg (nat_msg n) (name_msg parent);
         become empty_behv
       ).
\end{lstlisting}

まず、\texttt{factorial\_cont\_behv} と \texttt{factorial\_behv} を定義している。
これはそれぞれ何をかければいいかという自然数を保持するアクターと、それを作るアクターの振る舞いになっている。
The reason why these are defined using \texttt{CoFixpoint} is to define recursive function with no decreasing.
Coq's \texttt{Fixpoint}, used in definition of recursive function, imposes decreasing of size of argument so we cannot use \texttt{Fixpoint}.
次に \texttt{factorial\_system} という関数を定義している。
これはシステムを走らせるための関数になっている。

\section{semantics}
\label{sec:semantics}

In this section, we explain the formalization of operational semantics of the Actor model in Actario.
First, for the explanation of formalization of operational semantics, we describe \lstinline|name| type, \lstinline|actor| type, \lstinline|in_flight_message| type, and \lstinline|config| type.
And then, we explain how to formalize the operational semantics in Actario.

\subsection{Actor Name}
In Actario, actor name is defined as disjoint sum of the actor with no parent and the actor generated by another actor (Figure \ref{coq:name}).
We call the actor with no parent \textit{toplevel actor}.
This represents initial actors in the system.
And we call the actor generated by another actor \textit{generated actor}.
The name of a generated actor consists of the name of parent actor and the number that the parent actor generated so far.
We call the number \textit{generation number}.
To keep name uniqueness, we introduce generation number.
For more detail about name uniqueness, see section \ref{sec:uniqueness}.

%% Actario ではアクターの名前は、親がいないアクターと何らかのアクターから生成されたアクターに分けて定義する (図\ref{coq:name})。
%% 親がいないアクターは toplevel actor と呼ぶ。
%% これはアクターシステムに最初から存在するアクターを表す。
%% 何らかのアクターから生成されたアクターを generated actor と呼ぶ。
%% generated actor の名前は、親アクターの名前と、その親アクターが何番目に生成した子かという番号 (generation number) のペアとする。
%% これは動的に生成されうるアクターの名前の一意性を保つためである。

\begin{figure}[t]
  \begin{lstlisting}
    Inductive name : Set :=
    | toplevel : string -> name
    | generated : nat -> name -> name.
  \end{lstlisting}
  \caption{name}\label{coq:name}
\end{figure}


\subsection{actor}
We explain how \lstinline|actor| is defined in Actario.
Actor consists of its name, sequence of remaining actions, and next generation number to use in generating next child (Figure \ref{coq:actor}).
If remaining actions are only \textsf{become}, the actor is ready for receiving a message.
%% アクターは、自分自身名前、残りのアクションの列、次回アクターを生成する際に使う generation number の3つのレコード型である (図\ref{coq:actor})。
%% 残りのアクションの列が become のみの場合、このアクターはメッセージを受け取れる状態にあるということを表す。

\begin{figure}[t]
  \begin{lstlisting}
    Record actor := {
      actor_name : name;
      remaining_actions : actions;
      next_num : gen_number
    }.
  \end{lstlisting}
  \caption{actor}\label{coq:actor}
\end{figure}

\subsection{in flight message}
Next, we define \lstinline|in_flight_message| type which represents messages in flight in the configuration.
\lstinline|in_flight_message| consists of the name of destination, the name of sender, and the content of the message (Figure \ref{coq:inflight}).

宛先と送り主とメッセージの内容からなるレコード型である (図\ref{coq:inflight}))。
まだ受け取られていないメッセージを表す。
configuration に用いる。

\begin{figure}[t]
  \begin{lstlisting}
    Record in_flight_message := {
      to : name;
      from : name;
      content : message
    }.
  \end{lstlisting}
  \caption{in flight message}\label{coq:inflight}
\end{figure}

\subsection{configuration}
\textit{configuration} represents snapshot of the actor system.
configuration is used to formulate operational semantics of the Actor model.
In Actario, configuration consists of a list of actors and a list of messages in flight.

\begin{figure}[t]
  \begin{lstlisting}
    Record config := {
      in_flight_messages : list in_flight_message;
      actors : list actor
    }.
  \end{lstlisting}
  \caption{config}\label{coq:config}
\end{figure}


\subsection{label}
Actario formulates operational semantics of the Actor model as labeled transition system, so we define label (Figure \ref{coq:label}).
The explanations of each label are as follows.

%% Actario ではアクターモデルの操作的意味を labeled transition system として定式化するため、ラベルを定義する (図\ref{coq:label}))。
%% 以下にそれぞれの説明を示す。

\begin{description}[style=nextline,leftmargin=12pt,parsep=0pt]
\item[\texttt{Receive (to : name) (from : name) (content : message)}]
  This represents that the actor named \lstinline|to| receives the message \lstinline|content| sent from the actor named \lstinline|from|.
  %% \texttt{to} という名前のアクターが \texttt{from} という名前のアクターからのメッセージ \texttt{content} を受け取って遷移したということを表す。
\item[\texttt{Send (from : name) (to : name) (content : message)}]
  This represents that the actor named \lstinline|from| sends the message \lstinline|content| to the actor named \lstinline|to|.
  %% \texttt{from} という名前のアクターが \texttt{to} という名前のアクターに向けてメッセージ \texttt{content} を送ってシステムが遷移したということを表す。
\item[\texttt{New (child : name)}]
  This represents that the actor named \lstinline|child| is generated.
  %% \texttt{child} という名前の新しいアクターが生成されてシステムが遷移したということを表す。
\item[\texttt{Self (me : name)}]
  This represents that the actor named \lstinline|me| gets the name itself.
  %% \texttt{me} という名前のアクターが自分自身の名前を読みだしたということを表す。
\end{description}


\begin{figure}[t]
  \begin{lstlisting}
Inductive label :=
| Receive (to : name) (from : name) (content : message)
| Send (from : name) (to : name) (content : message)
| New (child : name)
| Self (me : name).
  \end{lstlisting}
  \caption{label}\label{coq:label}
\end{figure}


\subsection{semantics}

We formulate operational semantics of the Actor model as labeled transition system.
For later explanation, we define the symbols as shown in the figure \ref{fig:config}.
%% アクターモデルの操作的意味を configuration のラベル付き遷移システムとして定式化する。
%% これ以降用いる記号を図\ref{fig:config}のように定義する。

\begin{figure}[t]
  \begin{displaymath}
    \begin{array}{rclcl}
      c & \in & \textit{Configuration} & =   & \textit{Set(InFlight)} \times \textit{Set(Actor)} \\
      a & \in & \textit{Actor}  & =   & \textit{Name} \times \textit{Actions} \times \mathbb{N} \\
      n & \in & \textit{Name}   & ::= & \textsf{toplevel}(s) \mid \textsf{generated}(g, n) \\
      m & \in & \textit{Message} & =  & \textit{Name} + \textit{PrimVal} + \\
        &     &                 &     & \textit{Message} \times \cdots \times \textit{Message} \\
      i & \in & \textit{InFlight} & = & \textit{Name} \times \textit{Name} \times \textit{Message} \\
      b & \in & \textit{Behavior} & = & \textit{Message} \rightarrow \textit{Actions} \\
      \alpha & \in & \textit{Actions} & ::= & \textsf{send}(n, m, \alpha) \\
        &     &                 &   | & \textsf{new}(b, \kappa) \\
        &     &                 &   | & \textsf{self}(\kappa) \\
        &     &                 &   | & \textsf{become}(b) \\
      l & \in & \textit{Label}  & ::= & \textsf{Receive}(n, n, m) \\
        &     &                 &   | & \textsf{Send}(n, n, m) \\
        &     &                 &   | & \textsf{New}(n) \\
        &     &                 &   | & \textsf{Self}(n) \\
      \kappa & \in & \textit{Name} \rightarrow \textit{Actions} & & \\
      g & \in & \mathbb{N} & &
    \end{array}
  \end{displaymath}
  \caption{Configuration}\label{fig:config}
\end{figure}

The labeled transition system used in Actario is defined like figure \ref{fig:semantics}.
The explanations for each of transitions are the followings.
%% このラベル一つ一つに対応した意味論は図\ref{semantics}にある。
%% receive はメッセージ待ち状態にあるアクターが、自身に向けて送られたメッセージを受け取り、アクションの列を生成する遷移である。

\begin{description}[style=nextline,leftmargin=12pt,parsep=0pt]
\item[\textsc{Receive}]
  \textsc{Receive} is the transition for \textsf{Receive} label.
  The actor which is ready to receive a message, in other words, the actor whose remaining actions are only \textsf{become}, receives a message and generate new remaining actions by the behavior and the content of the message.
\item[\textsc{Send}]
  \textsc{Send} is the transition for \textsf{Send} label.
  The actor which want to send a message sends a message, and then the message is added into messages in flight.
\item[\textsc{New}]
  \textsc{New} is the transition for \textsf{New} label.
  An actor generates its child actor by the given behavior.
  And then, do the followings:
  \begin{itemize}
  \item The child actor is added into the configuration. The next generation number of child actor is 0.
  \item The next generation number of the parent actor increase by 1.
  \item The child actor is ready to receive a message.
  \end{itemize}
\item[\textsc{Self}]
  \textsc{Self} is the transition for \textsf{Self} label.
  An actor gets the self name and applies it to the continuation.
\end{description}

The definition in Actario is in Appendix \ref{app:lts}.

\begin{figure*}[t]
  \begin{displaymath}
    \begin{array}{rcll}
      (I \uplus \{(n_{\textrm{to}}, n_{\textrm{from}}, m)\}, A \cup \{(n_{\textrm{to}}, \textsf{become}(b), g)\}) &
      \overset{\textsf{Receive}(n_{\textrm{to}}, n_{\textrm{from}}, m)}{\leadsto} &
      (I, A \cup \{(n_{\textrm{to}}, b(m), g)\}) &
      \textsc{(Receive)} \\[1ex]

      (I, A \cup \{(n_{\textrm{from}}, \textsf{send}(n_{\textrm{to}}, m, \alpha), g)\}) &
      \overset{\textsf{Send}(n_{\textrm{from}}, n_{\textrm{to}}, m)}{\leadsto} &
      (I \uplus \{(n_{\textrm{to}}, n_{\textrm{from}}, m)\}, A \cup \{(n_{\textrm{from}}, \alpha, g)\}) &
      \textsc{(Send)} \\[1ex]

      (I, A \cup \{(n, \textsf{new}(b, \kappa), g)\}) &
      \overset{\textsf{New}(n')}{\leadsto} &
      (I, A \cup \{(n, \kappa(n'), g + 1), (n', \textsf{become}(b), 0)\}) & \\
      & & \hfill \textrm{where}\ n' := \textsf{generated}(g, n) &
      \textsc{(New)} \\[1ex]

      (I, A \cup \{(n, \textsf{self}(\kappa), g)\}) &
      \overset{\textsf{Self}(n)}{\leadsto} &
      (I, A \cup \{n, \kappa(n), g\}) &
      \textsc{(Self)}
    \end{array}
  \end{displaymath}
  \caption{labeled transition semantics}\label{fig:semantics}
\end{figure*}


\section{Acknowledgments}
This section is optional; it is a location for you
to acknowledge grants, funding, editing assistance and
what have you.  In the present case, for example, the
authors would like to thank Gerald Murray of ACM for
his help in codifying this \textit{Author's Guide}
and the \textbf{.cls} and \textbf{.tex} files that it describes.

%
% The following two commands are all you need in the
% initial runs of your .tex file to
% produce the bibliography for the citations in your paper.
\bibliographystyle{abbrv}
\bibliography{sigproc}  % sigproc.bib is the name of the Bibliography in this case
% You must have a proper ".bib" file
%  and remember to run:
% latex bibtex latex latex
% to resolve all references
%
% ACM needs 'a single self-contained file'!
%
%APPENDICES are optional
%\balancecolumns
\appendix
%Appendix A
\section{Headings in Appendices}
The rules about hierarchical headings discussed above for
the body of the article are different in the appendices.
In the \textbf{appendix} environment, the command
\textbf{section} is used to
indicate the start of each Appendix, with alphabetic order
designation (i.e. the first is A, the second B, etc.) and
a title (if you include one).  So, if you need
hierarchical structure
\textit{within} an Appendix, start with \textbf{subsection} as the
highest level. Here is an outline of the body of this
document in Appendix-appropriate form:
\subsection{Introduction}
\subsection{The Body of the Paper}
\subsubsection{Type Changes and  Special Characters}
\subsubsection{Math Equations}
\paragraph{Inline (In-text) Equations}
\paragraph{Display Equations}
\subsubsection{Citations}
\subsubsection{Tables}
\subsubsection{Figures}
\subsubsection{Theorem-like Constructs}
\subsubsection*{A Caveat for the \TeX\ Expert}
\subsection{Conclusions}
\subsection{Acknowledgments}
\subsection{Additional Authors}
This section is inserted by \LaTeX; you do not insert it.
You just add the names and information in the
\texttt{{\char'134}additionalauthors} command at the start
of the document.
\subsection{References}
Generated by bibtex from your ~.bib file.  Run latex,
then bibtex, then latex twice (to resolve references)
to create the ~.bbl file.  Insert that ~.bbl file into
the .tex source file and comment out
the command \texttt{{\char'134}thebibliography}.
% This next section command marks the start of
% Appendix B, and does not continue the present hierarchy
\section{More Help for the Hardy}
The sig-alternate.cls file itself is chock-full of succinct
and helpful comments.  If you consider yourself a moderately
experienced to expert user of \LaTeX, you may find reading
it useful but please remember not to change it.
%\balancecolumns % GM June 2007
% That's all folks!
\end{document}
