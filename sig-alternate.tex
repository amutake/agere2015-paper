% This is "sig-alternate.tex" V2.0 May 2012
% This file should be compiled with V2.5 of "sig-alternate.cls" May 2012
%
% This example file demonstrates the use of the 'sig-alternate.cls'
% V2.5 LaTeX2e document class file. It is for those submitting
% articles to ACM Conference Proceedings WHO DO NOT WISH TO
% STRICTLY ADHERE TO THE SIGS (PUBS-BOARD-ENDORSED) STYLE.
% The 'sig-alternate.cls' file will produce a similar-looking,
% albeit, 'tighter' paper resulting in, invariably, fewer pages.
%
% ----------------------------------------------------------------------------------------------------------------
% This .tex file (and associated .cls V2.5) produces:
%       1) The Permission Statement
%       2) The Conference (location) Info information
%       3) The Copyright Line with ACM data
%       4) NO page numbers
%
% as against the acm_proc_article-sp.cls file which
% DOES NOT produce 1) thru' 3) above.
%
% Using 'sig-alternate.cls' you have control, however, from within
% the source .tex file, over both the CopyrightYear
% (defaulted to 200X) and the ACM Copyright Data
% (defaulted to X-XXXXX-XX-X/XX/XX).
% e.g.
% \CopyrightYear{2007} will cause 2007 to appear in the copyright line.
% \crdata{0-12345-67-8/90/12} will cause 0-12345-67-8/90/12 to appear in the copyright line.
%
% ---------------------------------------------------------------------------------------------------------------
% This .tex source is an example which *does* use
% the .bib file (from which the .bbl file % is produced).
% REMEMBER HOWEVER: After having produced the .bbl file,
% and prior to final submission, you *NEED* to 'insert'
% your .bbl file into your source .tex file so as to provide
% ONE 'self-contained' source file.
%
% ================= IF YOU HAVE QUESTIONS =======================
% Questions regarding the SIGS styles, SIGS policies and
% procedures, Conferences etc. should be sent to
% Adrienne Griscti (griscti@acm.org)
%
% Technical questions _only_ to
% Gerald Murray (murray@hq.acm.org)
% ===============================================================
%
% For tracking purposes - this is V2.0 - May 2012

\documentclass{sig-alternate}

\usepackage{amssymb}
\usepackage{amsmath}
\usepackage{xcolor}
\usepackage{listings,lstcoq}
\usepackage{bcprules}
%\usepackage{courier}
%\usepackage{proof}
%\usepackage{hyperref}
%\usepackage[dvipdfmx]{graphicx}

%\newtheorem{definition}{Definition}
%\newtheorem{proposition}{Proposition}
%\newtheorem{lemma}{Lemma}
%\newtheorem{proof}{Proof}
\usepackage{luatexja-fontspec}

\lstdefinestyle{default}{
  breaklines=true,
  basicstyle=\ttfamily\normalsize,
  frame=tb,
  framesep=6pt,
  captionpos=b,
  language=coq
}
\lstset{
  style=default
}

\begin{document}
%
% --- Author Metadata here ---
\conferenceinfo{AGERE!@SPLASH}{Oct., 2015, Pittsburgh, PA, USA}
%\CopyrightYear{2007} % Allows default copyright year (20XX) to be over-ridden - IF NEED BE.
%\crdata{0-12345-67-8/90/01}  % Allows default copyright data (0-89791-88-6/97/05) to be over-ridden - IF NEED BE.
% --- End of Author Metadata ---

\title{Actario: A Framework for Reasoning About Actor Systems}
%\subtitle{[Extended Abstract]
%\titlenote{A full version of this paper is available as
%\textit{Author's Guide to Preparing ACM SIG Proceedings Using
%\LaTeX$2_\epsilon$\ and BibTeX} at
%\texttt{www.acm.org/eaddress.htm}}}
%
% You need the command \numberofauthors to handle the 'placement
% and alignment' of the authors beneath the title.
%
% For aesthetic reasons, we recommend 'three authors at a time'
% i.e. three 'name/affiliation blocks' be placed beneath the title.
%
% NOTE: You are NOT restricted in how many 'rows' of
% "name/affiliations" may appear. We just ask that you restrict
% the number of 'columns' to three.
%
% Because of the available 'opening page real-estate'
% we ask you to refrain from putting more than six authors
% (two rows with three columns) beneath the article title.
% More than six makes the first-page appear very cluttered indeed.
%
% Use the \alignauthor commands to handle the names
% and affiliations for an 'aesthetic maximum' of six authors.
% Add names, affiliations, addresses for
% the seventh etc. author(s) as the argument for the
% \additionalauthors command.
% These 'additional authors' will be output/set for you
% without further effort on your part as the last section in
% the body of your article BEFORE References or any Appendices.

\numberofauthors{2} %  in this sample file, there are a *total*
% of EIGHT authors. SIX appear on the 'first-page' (for formatting
% reasons) and the remaining two appear in the \additionalauthors section.
%
\author{
% You can go ahead and credit any number of authors here,
% e.g. one 'row of three' or two rows (consisting of one row of three
% and a second row of one, two or three).
%
% The command \alignauthor (no curly braces needed) should
% precede each author name, affiliation/snail-mail address and
% e-mail address. Additionally, tag each line of
% affiliation/address with \affaddr, and tag the
% e-mail address with \email.
%
% 1st. author
\alignauthor
Shohei Yasutake\\
       \affaddr{Department of Computer Science}\\
       \affaddr{Tokyo Institute of Technology}\\
       \affaddr{2-12-1 Ookayama, Meguroku, Tokyo}\\
       \affaddr{152-8552, Japan}\\
       \email{yasutake@psg.cs.titech.ac.jp}
% 2nd. author
\alignauthor
Takuo Watanabe\\
       \affaddr{Department of Computer Science}\\
       \affaddr{Tokyo Institute of Technology}\\
       \affaddr{2-12-1 Ookayama, Meguroku, Tokyo}\\
       \affaddr{152-8552, Japan}\\
       \email{takuo@acm.org}
}
% There's nothing stopping you putting the seventh, eighth, etc.
% author on the opening page (as the 'third row') but we ask,
% for aesthetic reasons that you place these 'additional authors'
% in the \additional authors block, viz.
\date{7 Aug. 2015}
% Just remember to make sure that the TOTAL number of authors
% is the number that will appear on the first page PLUS the
% number that will appear in the \additionalauthors section.

\maketitle
\begin{abstract}
The two main characteristics of the Actor model are asynchronous message passing and dynamic system topology. The latter relies on the on-the-fly creation of actor names that often complicates the formal treatment of systems described in the Actor model. In this paper, we introduce Actario, a formalization of the Actor model in Coq. Actario incorporates a name creation mechanism that is formally proven to generate a consistent set of actor names. The mechanism helps proper handling of names in modeling and reasoning about actor-based systems. Actario also provides a code extraction mechanism that generates Erlang programs.
\end{abstract}


% A category with the (minimum) three required fields
\category{F.3.2}{Logics and Meanings of Programs}%
{Specifying and Verifying and Reasoning about Programs}%
[Mechanical verification]

\terms{Actors, Formal Models}

\keywords{Actor Model, Formalization, Actario, Coq, Erlang}

\section{Introduction}
\label{sec:introduction}


The Actor model\cite{Agha:1986aa} is a kind of concurrent computation
model, in which a system is expressed as a collection of autonomous
computing entities called actors that communicate each other only with
asynchronous messages.
On receiving a message, an actor may (1)
send messages to other actors (or itself) whose names are known to the
sender, (2) create new actors and (3) change its behavior for the next
message.

Starting from the 1970s, the Actor model and its variations such as
concurrent objects\cite{Yonezawa:1986aa} have a long research
history. They are today regarded as popular high-level abstractions
for concurrent and parallel programming used in some industrial
strength language and libraries such as Erlang\cite{Erlang},
Scala\cite{Scala} and Akka\cite{Akka}.  Because of this situation,
establishing a mechanized formal verification method for actor-based
systems is a pressing issue.

Several methods and systems for formally verifying actor-based systems
have been presented recently. Rebeca\cite{Sirjani:2011aa} is a
modeling language that allows model-checking.  For deductive
verification using proof assistants, formalizations using
Athena\cite{Musser:2013aa} and Coq\cite{Garnock-Jones:2014aa} have
been presented.

A \emph{name}\footnote{The term \emph{mail address} is used in some other
  literature.} in the Actor model is a unique conceptual location
associated with each actor.  The concept of \emph{name uniqueness}
denotes that each name uniquely refers an actor and each actor should
be referred by a single name.  In the implementations of actor systems
including Erlang, Scala, and Akka, naming of actors is implicit; we
don't need to manually assign a fresh name to a newly created actor.
The name uniquness may be broken if the naming is explicit in complex
systems.  Implicit naming, however, might complicates the formal
treatment of actor-based systems. Thus, some formalization adopts
explicit naming.


In this paper, we propose Actario\cite{Actario}, a Coq framework for
implementing and verifying actor-based systems.  The framework (1)
supports Erlang-like notation for describing an actor system, (2)
allows verifying desired properties of the system using the proof
mechanism in Coq, and (3) generates executable Erlang code from the
system description.

To be close to realistic actor languages and libraries, we designed
Actario to support implicit naming. This is the main difference between
our formalization and formalizations using Athena\cite{Musser:2013aa}
or Coq\cite{Garnock-Jones:2014aa}. The naming mechanism behind the scene
is formally proved to satisfy the name uniqueness.  We also proved other
properties including the persistence of actors and messages.  The proof
scripts of these properties are available in the GitHub repository of
Actario\cite{Actario}.


% 形式的検証をおこなう上での問題
% 特にアクターの名前の生成,一貫性

%% 特にアクターの名前の
%% アクターの名前は必ず一意である必要がある。
%% アクターはシステムの進行に伴って動的に生成されるが、その中で fresh でない名前の生成が行われることはない。
%% fresh なアクターの名前の付け方には、~がある。

% ここで解決したい問題は何か(問題)
% * アクターモデルは並行計算のモデルとして real world でよく使われている
% * 一般的に並行システムの検証は難しい
% * アクターシステム (アクターモデルで記述されたアプリケーション) の検証を行いたい
% Actarioではどのようにして問題を解決しようとしているか(手法)
% 提案手法
% * 名前付けの方法とそれによる名前の一貫性の形式的証明
% それによってどのようなことが明らかになるか:
% * 普通のプログラミングと同様にモデル化できる
% * 名前を勝手につけられるので検証は難しくなる(?)

% Actarioの特徴
% * 普通のプログラミングと同様にシステムを記述できる
% * Coqによって検証もできる
%   + 実装しているシステム自身はCoqによって検証されている
%     - 名前の一貫性(済)
%     - アクターのpersistence (済)
%     - メッセージのpersistence (済)
% * Erlangプログラムを生成できる

The layout of the rest of this paper is as follows.
The next section describes the overview of Actario.
In Section~\ref{sec:semantics}, we give the operational semantics of the Actor model formalized in Actario.
Section~\ref{sec:uniqueness} outlines the proof of the uniqueness property on dynamically generated names.
In Section~\ref{sec:fairness}, we discuss fairness properties formalized in Actario.
The code extraction mechanism is described in Section~\ref{sec:extraction}.
Finally, Section~\ref{sec:relatedwork} overviews related work and Section~\ref{sec:conclusion} concludes the paper.

%% 2節では Actario で使うアクターのシンタックスを説明する。
%% 3節では Actario で用いるアクターモデルの意味論について説明する。
%% 4節では、Actario が用いる意味論において、動的に生成されるアクターの名前が一意になるという定理の証明を説明する。
%% 5節では Actario における fairness の定式化について説明し、
%% 6節で他のアクターモデルの形式化や並行・分散システムの検証用フレームワークとの差異を述べる。

\section{Syntax}

Actario では

以下のように定義する。

\begin{align*}
  configuration & ::= multiset of in\_flight\_message \bowtie set of actor \\
  actor & ::= Actor(name, actions, \mathbb{N}) \\
  name & ::= toplevel | generated(\mathbb{N}, name) \\
  actions & ::= name ! message; actions \\
  & \mid new behavior
\end{align*}


\subsection{Actor Name}
アクターの名前は、親がいないアクターと何らかのアクターから生成されたアクターに分けて定義する。
親がいないアクターは toplevel actor と呼ぶ。
これはアクターシステムに最初から存在するアクターを表す。
何らかのアクターから生成されたアクターを generated actor と呼ぶ。
generated actor の名前は、親アクターの名前と、その親アクターが何番目に生成した子かという番号 (generation number) のペアとする。
これは動的に生成されうるアクターの名前の一意性を保つためである。
基本的に Actario のプログラマは自分で名前を構築するようなことはしない。

\begin{lstlisting}
Inductive name : Set :=
| toplevel : string -> name
| generated : nat -> name -> name.
\end{lstlisting}

\subsection{message}
アクター間で取り交わされるメッセージの型を定義する。
Coq 内でメッセージを扱うため、Coq の型として定義する。
また、任意の型の値をメッセージとすることはできない。
これは、
1. これを導入すると name, 後述する actor, actions, behavior にメッセージの型パラメータを付与しなければいけなくなり、この後の証明が煩雑になる
2. ターゲットとして Erlang を想定しているので、Erlang のプリミティブをサポートするくらいでいい
3. この定義で十分
のためである。(理由が弱い)

\begin{lstlisting}
Inductive message : Set :=
| empty_msg : message
| name_msg : name -> message
| str_msg : string -> message
| nat_msg : nat -> message
| bool_msg : bool -> message
| tuple_msg : message -> message -> message.
\end{lstlisting}

\subsection{actions, behavior}
\texttt{actions} はアクターが行うアクションの列である。new, send, self, become がある。
behavior is a function from message to actions.

\begin{lstlisting}
CoInductive actions : Type :=
| new : behavior -> (name -> actions) -> actions
| send : name -> message -> actions -> actions
| self : (name -> actions) -> actions
| become : behavior -> actions
with behavior : Type :=
| receive : (message -> actions) -> behavior.
\end{lstlisting}

actions のそれぞれのコンストラクタを説明する。
\texttt{new : behavior -> (name -> actions) -> actions} は、新しいアクターを生成するアクションである。
引数として与えられたある behavior を initial behavior として新しいアクターを生成し、その名前をアクションの継続に渡す。
\texttt{send : name -> message -> actions -> actions} は、指定した名前のアクターにメッセージを送るアクションである。
第一引数に指定した名前のアクターに、第二引数で指定したメッセージを送り、第三引数で指定したその後のアクションを実行する、という意味である。
\texttt{self : (name -> actions) -> actions} は自分自身の名前を得るアクションである。
自分の名前を第一引数で与えられた継続に渡し、残りのアクションを得る。
\texttt{become : behavior -> actions} は次のメッセージに対して指定した振る舞いで処理する。
また、次のメッセージの待ち状態になることも表し、これ以降アクションは取れないようになっている。

actions と behavior は co-inductive types として定義される。
これは become で現在の behavior を指定したり、ある他の behavior と相互再帰するようなことがあるからである。
Coq ではこのような、コンストラクタの深さが小さくならない再帰は通常の inductive definition では書けなくなっている。

また、以下のように notation をつけている。

\begin{lstlisting}
Notation ``n '<-' 'new' behv ; cont'' := (new behv (fun n => cont)) (at level 0, cont at level 10).
Notation ``n '!' m ';' a'' := (send n m a) (at level 0, a at level 10).
Notation ``me '<-' 'self' ';' cont'' := (self (fun me => cont)) (at level 0, cont at level 10).
\end{lstlisting}

\texttt{n <- new behv; cont} は \texttt{new behv (fun n => cont)} に変換され、
\texttt{n ! msg; cont} は \texttt{send n msg cont} に変換され、
\texttt{me <- self; cont} は \texttt{self (fun me => cont)} に変換される。

\subsection{actor}
アクターは、自分自身名前、残りのアクションの列、次回アクターを生成する際に使う generation number の3つのレコード型である。
残りのアクションの列が become のみの場合、このアクターはメッセージを受け取れる状態にあるということを表す。

\begin{lstlisting}
Record actor := {
  actor_name : name;
  remaining_actions : actions;
  next_num : gen_number
}.
\end{lstlisting}

\subsection{in flight message}
宛先と送り主とメッセージの内容からなるレコード型である。
まだ受け取られていないメッセージを表す。
configuration に用いる。

\begin{lstlisting}
Record in_flight_message := {
  to : name;
  from : name;
  content : message
}.
\end{lstlisting}

\subsection{configuration}
configuration はアクターシステムの現在のスナップショットを表す。
configuration はアクターモデルの意味論を定義する際に用いるものである。
Actario では configuration を actor の列と in\_flight\_message の列のペアとして定義する。

\begin{lstlisting}
Record config := {
  in_flight_messages : seq in_flight_message;
  actors : seq actor
}.
\end{lstlisting}

\footnote{seq は ssreflect で定義されているもので、Coq の list の notation。}

\subsection{Example: factorial system}
例として階乗を計算するアクターシステムを定義する。
このシステムは、与えられた自然数 n に対して、n の階乗を計算した結果を、与えられた名前のアクターに対して送信するシステムである。
次に何をかけるかという情報を持った (continuation を持った) アクターを作るシステムになっている。\ref{agha:2004}

\begin{lstlisting}
CoFixpoint factorial_cont_behv (val : nat) (cust : name) : behavior :=
  receive (fun msg =>
         match msg with
           | nat_msg arg =>
             cust ! nat_msg (val * arg);
             become empty_behv
           | _ => become (factorial_cont_behv val cust)
         end).

CoFixpoint factorial_behv : behavior :=
  receive (fun msg =>
         match msg with
           | tuple_msg (nat_msg 0) (name_msg cust) =>
             cust ! nat_msg 1;
             become factorial_behv
           | tuple_msg (nat_msg (S n)) (name_msg cust) =>
             cont <- new (factorial_cont_behv (S n) cust);
             me <- self;
             me ! tuple_msg (nat_msg n) (name_msg cont);
             become factorial_behv
           | _ => become factorial_behv
         end).

Definition factorial_system (n : nat) (parent : name) : config :=
  init "factorial" (
         x <- new factorial_behv;
         x ! tuple_msg (nat_msg n) (name_msg parent);
         become empty_behv
       ).
\end{lstlisting}

まず、\texttt{factorial\_cont\_behv} と \texttt{factorial\_behv} を定義している。
これはそれぞれ何をかければいいかという自然数を保持するアクターと、それを作るアクターの振る舞いになっている。
The reason why these are defined using \texttt{CoFixpoint} is to define recursive function with no decreasing.
Coq's \texttt{Fixpoint}, used in definition of recursive function, imposes decreasing of size of argument so we cannot use \texttt{Fixpoint}.
次に \texttt{factorial\_system} という関数を定義している。
これはシステムを走らせるための関数になっている。

\section{semantics}
\label{sec:semantics}

In this section, we explain the formalization of operational semantics of the Actor model in Actario.
First, for the explanation of formalization of operational semantics, we describe \lstinline|name| type, \lstinline|actor| type, \lstinline|in_flight_message| type, and \lstinline|config| type.
And then, we explain how to formalize the operational semantics in Actario.

\subsection{Actor Name}
In Actario, actor name is defined as disjoint sum of the actor with no parent and the actor generated by another actor (Figure \ref{coq:name}).
We call the actor with no parent \textit{toplevel actor}.
This represents initial actors in the system.
And we call the actor generated by another actor \textit{generated actor}.
The name of a generated actor consists of the name of parent actor and the number that the parent actor generated so far.
We call the number \textit{generation number}.
To keep name uniqueness, we introduce generation number.
For more detail about name uniqueness, see section \ref{sec:uniqueness}.

%% Actario ではアクターの名前は、親がいないアクターと何らかのアクターから生成されたアクターに分けて定義する (図\ref{coq:name})。
%% 親がいないアクターは toplevel actor と呼ぶ。
%% これはアクターシステムに最初から存在するアクターを表す。
%% 何らかのアクターから生成されたアクターを generated actor と呼ぶ。
%% generated actor の名前は、親アクターの名前と、その親アクターが何番目に生成した子かという番号 (generation number) のペアとする。
%% これは動的に生成されうるアクターの名前の一意性を保つためである。

\begin{figure}[t]
  \begin{lstlisting}
    Inductive name : Set :=
    | toplevel : string -> name
    | generated : nat -> name -> name.
  \end{lstlisting}
  \caption{name}\label{coq:name}
\end{figure}


\subsection{actor}
We explain how \lstinline|actor| is defined in Actario.
Actor consists of its name, sequence of remaining actions, and next generation number to use in generating next child (Figure \ref{coq:actor}).
If remaining actions are only \textsf{become}, the actor is ready for receiving a message.
%% アクターは、自分自身名前、残りのアクションの列、次回アクターを生成する際に使う generation number の3つのレコード型である (図\ref{coq:actor})。
%% 残りのアクションの列が become のみの場合、このアクターはメッセージを受け取れる状態にあるということを表す。

\begin{figure}[t]
  \begin{lstlisting}
    Record actor := {
      actor_name : name;
      remaining_actions : actions;
      next_num : gen_number
    }.
  \end{lstlisting}
  \caption{actor}\label{coq:actor}
\end{figure}

\subsection{in flight message}
Next, we define \lstinline|in_flight_message| type which represents messages in flight in the configuration.
\lstinline|in_flight_message| consists of the name of destination, the name of sender, and the content of the message (Figure \ref{coq:inflight}).

宛先と送り主とメッセージの内容からなるレコード型である (図\ref{coq:inflight}))。
まだ受け取られていないメッセージを表す。
configuration に用いる。

\begin{figure}[t]
  \begin{lstlisting}
    Record in_flight_message := {
      to : name;
      from : name;
      content : message
    }.
  \end{lstlisting}
  \caption{in flight message}\label{coq:inflight}
\end{figure}

\subsection{configuration}
\textit{configuration} represents snapshot of the actor system.
configuration is used to formulate operational semantics of the Actor model.
In Actario, configuration consists of a list of actors and a list of messages in flight.

\begin{figure}[t]
  \begin{lstlisting}
    Record config := {
      in_flight_messages : list in_flight_message;
      actors : list actor
    }.
  \end{lstlisting}
  \caption{config}\label{coq:config}
\end{figure}


\subsection{label}
Actario formulates operational semantics of the Actor model as labeled transition system, so we define label (Figure \ref{coq:label}).
The explanations of each label are as follows.

%% Actario ではアクターモデルの操作的意味を labeled transition system として定式化するため、ラベルを定義する (図\ref{coq:label}))。
%% 以下にそれぞれの説明を示す。

\begin{description}[style=nextline,leftmargin=12pt,parsep=0pt]
\item[\texttt{Receive (to : name) (from : name) (content : message)}]
  This represents that the actor named \lstinline|to| receives the message \lstinline|content| sent from the actor named \lstinline|from|.
  %% \texttt{to} という名前のアクターが \texttt{from} という名前のアクターからのメッセージ \texttt{content} を受け取って遷移したということを表す。
\item[\texttt{Send (from : name) (to : name) (content : message)}]
  This represents that the actor named \lstinline|from| sends the message \lstinline|content| to the actor named \lstinline|to|.
  %% \texttt{from} という名前のアクターが \texttt{to} という名前のアクターに向けてメッセージ \texttt{content} を送ってシステムが遷移したということを表す。
\item[\texttt{New (child : name)}]
  This represents that the actor named \lstinline|child| is generated.
  %% \texttt{child} という名前の新しいアクターが生成されてシステムが遷移したということを表す。
\item[\texttt{Self (me : name)}]
  This represents that the actor named \lstinline|me| gets the name itself.
  %% \texttt{me} という名前のアクターが自分自身の名前を読みだしたということを表す。
\end{description}


\begin{figure}[t]
  \begin{lstlisting}
Inductive label :=
| Receive (to : name) (from : name) (content : message)
| Send (from : name) (to : name) (content : message)
| New (child : name)
| Self (me : name).
  \end{lstlisting}
  \caption{label}\label{coq:label}
\end{figure}


\subsection{semantics}

We formulate operational semantics of the Actor model as labeled transition system.
For later explanation, we define the symbols as shown in the figure \ref{fig:config}.
%% アクターモデルの操作的意味を configuration のラベル付き遷移システムとして定式化する。
%% これ以降用いる記号を図\ref{fig:config}のように定義する。

\begin{figure}[t]
  \begin{displaymath}
    \begin{array}{rclcl}
      c & \in & \textit{Configuration} & =   & \textit{Set(InFlight)} \times \textit{Set(Actor)} \\
      a & \in & \textit{Actor}  & =   & \textit{Name} \times \textit{Actions} \times \mathbb{N} \\
      n & \in & \textit{Name}   & ::= & \textsf{toplevel}(s) \mid \textsf{generated}(g, n) \\
      m & \in & \textit{Message} & =  & \textit{Name} + \textit{PrimVal} + \\
        &     &                 &     & \textit{Message} \times \cdots \times \textit{Message} \\
      i & \in & \textit{InFlight} & = & \textit{Name} \times \textit{Name} \times \textit{Message} \\
      b & \in & \textit{Behavior} & = & \textit{Message} \rightarrow \textit{Actions} \\
      \alpha & \in & \textit{Actions} & ::= & \textsf{send}(n, m, \alpha) \\
        &     &                 &   | & \textsf{new}(b, \kappa) \\
        &     &                 &   | & \textsf{self}(\kappa) \\
        &     &                 &   | & \textsf{become}(b) \\
      l & \in & \textit{Label}  & ::= & \textsf{Receive}(n, n, m) \\
        &     &                 &   | & \textsf{Send}(n, n, m) \\
        &     &                 &   | & \textsf{New}(n) \\
        &     &                 &   | & \textsf{Self}(n) \\
      \kappa & \in & \textit{Name} \rightarrow \textit{Actions} & & \\
      g & \in & \mathbb{N} & &
    \end{array}
  \end{displaymath}
  \caption{Configuration}\label{fig:config}
\end{figure}

The labeled transition system used in Actario is defined like figure \ref{fig:semantics}.
The explanations for each of transitions are the followings.
%% このラベル一つ一つに対応した意味論は図\ref{semantics}にある。
%% receive はメッセージ待ち状態にあるアクターが、自身に向けて送られたメッセージを受け取り、アクションの列を生成する遷移である。

\begin{description}[style=nextline,leftmargin=12pt,parsep=0pt]
\item[\textsc{Receive}]
  \textsc{Receive} is the transition for \textsf{Receive} label.
  The actor which is ready to receive a message, in other words, the actor whose remaining actions are only \textsf{become}, receives a message and generate new remaining actions by the behavior and the content of the message.
\item[\textsc{Send}]
  \textsc{Send} is the transition for \textsf{Send} label.
  The actor which want to send a message sends a message, and then the message is added into messages in flight.
\item[\textsc{New}]
  \textsc{New} is the transition for \textsf{New} label.
  An actor generates its child actor by the given behavior.
  And then, do the followings:
  \begin{itemize}
  \item The child actor is added into the configuration. The next generation number of child actor is 0.
  \item The next generation number of the parent actor increase by 1.
  \item The child actor is ready to receive a message.
  \end{itemize}
\item[\textsc{Self}]
  \textsc{Self} is the transition for \textsf{Self} label.
  An actor gets the self name and applies it to the continuation.
\end{description}

The definition in Actario is in Appendix \ref{app:lts}.

\begin{figure*}[t]
  \begin{displaymath}
    \begin{array}{rcll}
      (I \uplus \{(n_{\textrm{to}}, n_{\textrm{from}}, m)\}, A \cup \{(n_{\textrm{to}}, \textsf{become}(b), g)\}) &
      \overset{\textsf{Receive}(n_{\textrm{to}}, n_{\textrm{from}}, m)}{\leadsto} &
      (I, A \cup \{(n_{\textrm{to}}, b(m), g)\}) &
      \textsc{(Receive)} \\[1ex]

      (I, A \cup \{(n_{\textrm{from}}, \textsf{send}(n_{\textrm{to}}, m, \alpha), g)\}) &
      \overset{\textsf{Send}(n_{\textrm{from}}, n_{\textrm{to}}, m)}{\leadsto} &
      (I \uplus \{(n_{\textrm{to}}, n_{\textrm{from}}, m)\}, A \cup \{(n_{\textrm{from}}, \alpha, g)\}) &
      \textsc{(Send)} \\[1ex]

      (I, A \cup \{(n, \textsf{new}(b, \kappa), g)\}) &
      \overset{\textsf{New}(n')}{\leadsto} &
      (I, A \cup \{(n, \kappa(n'), g + 1), (n', \textsf{become}(b), 0)\}) & \\
      & & \hfill \textrm{where}\ n' := \textsf{generated}(g, n) &
      \textsc{(New)} \\[1ex]

      (I, A \cup \{(n, \textsf{self}(\kappa), g)\}) &
      \overset{\textsf{Self}(n)}{\leadsto} &
      (I, A \cup \{n, \kappa(n), g\}) &
      \textsc{(Self)}
    \end{array}
  \end{displaymath}
  \caption{labeled transition semantics}\label{fig:semantics}
\end{figure*}

\section{Name Uniqueness}
\label{sec:uniqueness}
%% \note{なぜこれが証明されていなければいけないのか、これが証明されていることによってどういうことが言えるのか}
In programming languages or libraries providing the Actor model such as Erlang or Akka,
the system automatically generates actors with fresh names without specifying the name explicitly by the programmer.
In Actario, the proposition that all actor names in the configuration are not duplicate by any transitions is proven.

%% アクターモデルを提供する言語やライブラリ、例えば Erlang や Akka では、アクターを生成する際にプログラマが名前を指定せずとも fresh な名前を生成してくれる。
%% また、アクターはシステムの進行中に動的に生成されうるものなので、動的に生成されうるアクターの名前が常に一意になることが重要である。
%% Actario では、Actario の意味論において、ある制限を満たした初期状態からの任意の遷移でアクターの名前が衝突しないことを証明した。

To prove, we define an invariant about actor names preserved between any transitions. It is named \textit{trans invariant}.
The trans invariant consists of the following three predicates for configuration.
%% システムの遷移において満たされる、名前についての性質 trans invariant を定義し、trans invariant が確かに遷移の間で保存されることを証明、そして trans invariant が成り立てば名前の一意性が成り立つことの証明、初期状態が

\begin{displaymath}
  \begin{array}{l}
    \texttt{trans\_invariant}(c) := \\
    \quad \texttt{chain}(c) \wedge \texttt{gen\_fresh}(c) \wedge \texttt{no\_dup}(c)
  \end{array}
\end{displaymath}

The brief explanations of \texttt{chain}, \texttt{gen\_fresh}, and \texttt{no\_dup} are followings:

\begin{description}[style=nextline,leftmargin=12pt,parsep=0pt]
\item[\texttt{chain}]
  For each actor in the configuration, if the actor is generated by another actor, then the parent actor is also in the configuration.
\item[\texttt{gen\_fresh}]
  For each actor in the configuration, actor name genereted by the actor in the next is fresh.
\item[\texttt{no\_dup}]
  For all actor name in the configuration are unique.
\end{description}

\subsection{functions}

Before starting the explanation and the proof, we define some functions used in this section.

\begin{description}[style=nextline,leftmargin=12pt,parsep=0pt]
\item[\texttt{actors} $: \textit{Configuration} \rightarrow \textit{Set(Actor)}$]
  \texttt{actors} returns the set of actors in the given configuration.
\item[\texttt{parent} $: \textit{Actor} \rightarrow \textit{Actor}$]
  \texttt{parent} returns the parent actor of the given actor.
  If the given actor is toplevel actor, the function returns nothing. % null?
\item[\texttt{gen\_number} $: \textit{Actor} \rightarrow \mathbb{N}$]
  \texttt{gen\_number} returns generated number of the name of the given actor.
  If the given actor is toplevel actor, the function returns nothing.
\item[\texttt{next\_number} $: \textit{Actor} \rightarrow \mathbb{N}$]
  \texttt{next\_number} returns next generation number of the given actor.
\item[\texttt{name} $: \textit{Actor} \rightarrow \textit{Name}$]
  \texttt{name} returns the name of the given actor.
\item[\texttt{names} $: \textit{Set(Actor)} \rightarrow \textit{Set(Name)}$]
  \texttt{names} returns names of the given set of actors.
\end{description}

\subsection{chain}
We define an predicate of configuration, called \texttt{chain}.
\texttt{chain} is the predicate that, for each actor in the given configuration, if it is generated by another actor, the parent actor is also in the configuration.
\texttt{chain} is defined as the following.

\begin{displaymath}
  \begin{array}{l}
    \texttt{chain}(c) := \\
    \quad \forall a \in \texttt{actors}(c), \exists p, p = \texttt{parent}(a) \Rightarrow p \in \texttt{actors}(c)
  \end{array}
\end{displaymath}

Then, we can prove \textit{chain preservation property} that chain is preserved between any transitions.
The proof is by case analysis on the label.
\texttt{chain} is decided by only actor names, and the transition which have a possibility to change the names in the configuration is only \textsc{New} transition.
Therefore, we consider only the case of \textsc{New} transition.

\begin{lemma}{chain preservation}
\begin{displaymath}
  \begin{array}{l}
    \forall c, c' \in \textit{Configuration}, \forall l \in \textit{Label}, \\
    \quad \texttt{chain}(c) \wedge c \overset{l}{\leadsto} c' \Rightarrow \texttt{chain}(c')
  \end{array}
\end{displaymath}
\end{lemma}

\subsection{gen\_fresh}
We define \texttt{gen\_fresh} predicate that, for each actor in the configuration, the name of its child is always fresh.
The definition of \texttt{gen\_fresh} is complicated a little.
We translate the proposition that next generated name is fresh to the following.

\begin{displaymath}
  \begin{array}{l}
    \texttt{gen\_fresh}(c) := \\
    \quad \forall a \in \texttt{actors}(c), \exists p, p = \texttt{parent}(a) \wedge \quad \quad p \in \texttt{actors}(c) \Rightarrow \\
    \texttt{gen\_number}(a) < \texttt{next\_number}(p)
  \end{array}
\end{displaymath}


It is guaranteed that the actor name generated in the next is fresh if satisfying \texttt{gen\_fresh} predicate by the relation of next generation numbers and actor names. %% For each actor in the configuration, if its parent is in the configuration, the next generation number of the parent actor is greater than the generation number of the name of the child actor.
However, the actor name generated after the next is not always fresh name.
For example, if there are two actors ($A$ and $B$) that have the same name and the same next generation number and actor $A$ generates a child actor and actor $B$ generates a child actor, although \texttt{gen\_fresh} holds, these child actors have the same name.
Furthermore, if the parent of the actor $A$ does not exist in the configuration and the parent of the parent exists in the configuration, and the parent of the parent actor generates an actor and it also generates an actor, then the name is possible to have the same as $A$'s one.

%% つまり、あるアクターについて、システム内に親アクターがいる場合、親アクターが次に生成する番号は自分の番号よりも大きい、ということにより、次に生成するアクターの名前が被らないようになっている。
%% ただし、次に生成するアクターの名前は fresh でもその次に生成するアクターは fresh ではないこともある。
%% 例えば、同じ名前でかつ次の generation number も同じという2つのアクターがいた場合、まず片方のアクターが生成するアクターの名前は fresh だが、その次にもう片方のアクターがアクターを生成したとすると、名前が被ってしまう。
%% また、親アクターがシステム内に存在せずに、親の親は存在しているという場合、親の親が次に生成するアクターの名前は被らないが、その子アクターが次に生成する名前は被ってしまう可能性がある。(図?)

Thus, to prove \textit{gen fresh preservation} proposition that \texttt{gen\_fresh} is preserved between transitions, it is necessary to use \texttt{chain} and \texttt{no\_dup} as hypotheses.
%% The proof is by ...
%% 以上のように gen\_fresh だけでは gen\_fresh を導けないので、gen\_fresh の証明には chain と no\_dup の性質が必要になる。

\begin{lemma}{gen fresh preservation}
\begin{displaymath}
  \begin{array}{l}
    \forall c, c' \in \textit{Configuration}, \forall l \in \textit{Label}, \\
    \quad \texttt{chain}(c) \wedge \texttt{gen\_fresh}(c) \wedge \texttt{no\_dup}(c) \wedge c \overset{l}{\leadsto} c' \Rightarrow \\
    \quad \texttt{gen\_fresh}(c')
  \end{array}
\end{displaymath}
\end{lemma}

\subsection{no\_dup}
We define \texttt{no\_dup} predicate that all actor names in the given configuration are unique.
This is the property we have to prove.
\texttt{no\_dup} is defined as the following.

\begin{displaymath}
  \begin{array}{l}
    \texttt{no\_dup}(c) := \\
    \quad \forall a \in \texttt{actors}(c), \texttt{name}(a) \notin
    \texttt{names}(\texttt{actors}(c) \setminus \{a\})
  \end{array}
\end{displaymath}

We proved \textit{no dup preservation} property defined as the following.
It represents that if the actor names in the configuration is not duplicate and the next generated actor name is fresh, then \texttt{no\_dup} holds in the next configuration.

\begin{lemma}{no dup preservation}
\begin{displaymath}
  \begin{array}{l}
    \forall c, c' \in \textit{Configuration}, \forall l \in \textit{Label}, \\
    \quad \texttt{gen\_fresh}(c) \wedge \texttt{no\_dup}(c) \wedge c \overset{l}{\leadsto} c' \Rightarrow \texttt{no\_dup}(c')
  \end{array}
\end{displaymath}
\end{lemma}

\subsection{uniqueness}
Then, we start to prove name uniqueness.
First, we prove trans invariant preservation that trans invariant is preserved between transitions.
This is obvious by chain preservation, gen fresh preservation and no dup preservation.
\begin{lemma}{trans invariant preservation}
  \begin{displaymath}
    \begin{array}{l}
      \forall c, c' \in \textit{Configuration}, \forall l \in \textit{Label}, \\
      \quad \texttt{trans\_invariant}(c) \wedge c \overset{l}{\leadsto} c' \Rightarrow \texttt{trans\_invariant}(c')
    \end{array}
  \end{displaymath}
\end{lemma}

Next, we prove that if trans invariant holds in initial configuration, trans invariant holds after arbitrary transitions.

%% 次に初期状態について trans\_invariant が成り立っていれば、任意回の遷移後も trans\_invariant が成り立つということをを証明する。

\begin{lemma}{trans invariant preservation star}
  \begin{displaymath}
    \begin{array}{l}
      \forall c, c' \in \textit{Configuration}, \forall l \in \textit{Label}, \\
      \quad \texttt{trans\_invariant}(c) \wedge c \overset{l}{\leadsto\star} c' \Rightarrow \texttt{trans\_invariant}(c')
    \end{array}
  \end{displaymath}
\end{lemma}
$c \overset{l}{\leadsto\star} c'$ represents reflexive transitive closure of transition.
The proof is by induction of reflexive transitive closure of transition and trans invariant preservation.

Finally, we can prove name uniqueness.
\begin{theorem}{name uniqueness}
  \begin{displaymath}
    \begin{array}{l}
      \forall c, c' \in \textit{Configuration}, \forall l \in \textit{Label}, \\
      \quad \texttt{trans\_invariant}(c) \wedge c \overset{l}{\leadsto\star} c' \Rightarrow \texttt{no\_dup}(c')
    \end{array}
  \end{displaymath}
\end{theorem}
This is obvious by trans invariant preservation star because \texttt{no\_dup} is in \texttt{trans\_invariant}.

\section{Fairness}

\textsf{fairness} is a property that reception of a message does not delay infinitely.
There are two variants of fairness property, weak fairness and strong fairness.
Weak fairness is that if an actor is infinitely always ready to receive the message, the message is eventually received.
Strong fairness is that if an actor is infinitely often ready to receive the message, the message is eventually received.
The Actor model satisfies strong fairness.
In this section, we define strong fairness in Actario.

\subsection{Transition Path}
Generally, fairness is represented in using operators of temporal logic.
We have to encode temporal logic because Coq does not support temporal logic.
We use transition path, which represents transition sequence of configuration, to define fairness as predicate of transition path.
This method is used in Appl$\pi$ \cite{Affeldt200817}.

We define transition path as function of $\mathbb{N}$ to \texttt{option config}.
In this definition, $\mathbb{N}$ represents the number of transitions from initial configuration and the reason why return value is wrapped with \texttt{option} is that it may be no more transitions.
% つまり、i 番目の configuration から遷移先がない場合は、i + 1 番目以降は None になる。

\begin{lstlisting}
Definition path := nat -> option config.
\end{lstlisting}

And we define the predicate that the given path is correct transition path.
%% また、与えられた transition path が確かに transition path になっているか、という述語を定義する。すべての index n について、n 番目の configuration が存在するならば、n + 1 番目の configuration が存在するならそれは遷移できるものか、それ以上遷移できない。n 番目の configuration が存在しないならば、その次の configuration も存在しない、という意味である。

\begin{lstlisting}
Definition is_transition_path (p : path) : Prop :=
  forall n,
    (forall c, p n = Some c ->
               (exists c' l, p (S n) = Some c' /\ c ~(l)~> c') \/
               p (S n) = None) /\
    (p n = None -> p (S n) = None).
\end{lstlisting}

\subsection{enabled}
We define the predicate that the transition from the given configuration with the given label is possible, called \texttt{enabled}.
\texttt{enabled} is defined as there exists a configuration after transition from the configuration with the label.
In Actario, \texttt{enabled} is defined as follows.
%% ある遷移ができる状態にある、ということを enabed と呼ぶ。これは、ある configuration からあるラベルによって遷移した先の configuration が存在する、と定義する。

\begin{lstlisting}
Definition enabled (c : config) (l : label) : Prop :=
  exists c', c ~(l)~> c'.
\end{lstlisting}

\subsection{infinitely often enabled}
We define the predicate that the transition is infinitely often enabled in the transition path.
%% これは、すべての index n について、n 番目の configuration があるラベルによって遷移が可能ならば、その先そのラベルによって遷移が可能になる configuration が存在する、と定義する。

\begin{lstlisting}
Definition infinitely_often_enabled (l : label) (p : path) : Prop :=
  forall n c, p n = Some c ->
              enabled c l ->
              exists m c', m > n /\
                           p m = Some c' /\
                           enabled c' l.
\end{lstlisting}


\subsection{eventually processed}
We define \texttt{eventually processed} that is the predicate of label and transition path.
It represents that the transition with the label is processed eventually in the path.
It is defined as follows.

\begin{lstlisting}
Definition eventually_processed (l : label) (p : path) : Prop :=
  exists n c c',
    p n = Some c /\ p (S n) = Some c' /\ c ~(l)~> c'.
\end{lstlisting}


\subsection{Definition of fairness}
Then we can define \texttt{fairness} predicate for transition path.
For the given transition path and for each label, if \texttt{infinitely often enabled} holds, then \texttt{eventually processed} holds.
\texttt{is postfix of} predicate is used for representing 'infinite'.
If \texttt{is postfix of} is not used, the transition may not be processed after the transition is processed although the transition is processed in whole the path.
To prevent it, if \texttt{inifinitely often enabled} holds then \texttt{eventually processed} holds for arbitrary postfix path by using \texttt{is postfix path}.

\begin{lstlisting}
Definition is_postfix_of (p' p : path) : Prop :=
  exists n, (forall m, p' m = p (m + n)).

Definition fairness : Prop :=
  forall p p', is_postfix_of p' p ->
               (forall l,
                  infinitely_often_enabled l p' ->
                  eventually_processed l p').
\end{lstlisting}

\section{Related Work}
\label{sec:relatedwork}

Appl\(\pi\) is a Coq library for modeling and verifying concurrent programs \cite{Affeldt200817}.
Actario is very inspired by Appl\(\pi\), for example, the definition of fairness, continuation passing style in \texttt{actions} and framework design.
The main difference of Appl\(\pi\) and Actario is that Appl\(\pi\) adopts \(\pi\)-calculus for its concurrent computation basic, but Actario adopts the Actor model for its concurrent computation basic.

Musser and Varela are formalized the Actor model in the Athena theorem prover \cite{Athena}\cite{Musser:2013aa}. % In this paper, transition path is defined as sequence of labels.
In this paper, name uniqueness is proven.
However, a programmer has to name new actors explicitly.
Therefore, a programmer has to select a fresh name. It is difficult to give always fresh name in complex system.
In addition, it is impossible to run program built in the formalization, while Actario can by extraction.
%% これは transition path を遷移のラベルの列としている。
%% この形式化では遷移の間で名前の一意性が証明されているが、 % note: creating, trans-create, unique-ids-persistence in transition.ath
%% この形式化を用いてプログラミングする際には名前をプログラマが明示的に与えなければならないので、アクターを生成するときには名前が重複しないように注意深く設定しなければならない。
%% また、Actario の方がより実際のプログラミングを行うに近いプログラミングができる。
%% Extraction はない。

Verdi is a framework for constructing and verifying fault-tolerant distributed systems \cite{Verdi}.
A system assumed no network failure is converted to the system which tolerates dropping packets, duplication of packets, and machine failure.
One of the purpose of Actario is also to build and verify fault-tolerant distributed systems.
We will introduce \textit{supervisor} mechanism to achieve building fault-tolerant systems generally used in Erlang and Akka.
\note{Supervisor についての説明はいるかどうか}
%% Supervisor is used for fault-tolerance and rapid recovery in the system, introduced in Erlang, Akka, and so on.

%% operate correctly; preserving the properties of the system.
%% 故障がまったくない意味論上で作ったシステムを、メッセージのドロップや重複、マシンの以上終了などを含む意味論上でも正常に動作 (ここでいう正常に動作とは、そのシステムについて成り立っていてほしい性質が成り立ついるということ) するシステムへと変換する仕組みが備わっている。
%% Actario の目標の一つも分散システムの検証で、Actario では Erlang や Akka で採用されているような Supervisor を使った耐障害性のあるシステムに対しての検証を目指している。
%% Supervisor は Erlang や Akka で取り入れられている考え方で、

Tony Garnock-Jones, Sam Tobin-Hochstadt, and Matthias Felleisen give a formalization of the Actor model using Coq \cite{Garnock-Jones:2014aa}.
In this paper, operational semantics is formalized so that transition is decidable.
Due to this, it is difficult to apply the formalization to realistic concurrent systems.

%
% The following two commands are all you need in the
% initial runs of your .tex file to
% produce the bibliography for the citations in your paper.
\bibliographystyle{abbrv}
\bibliography{sigproc}  % sigproc.bib is the name of the Bibliography in this case
% You must have a proper ".bib" file
%  and remember to run:
% latex bibtex latex latex
% to resolve all references
%
% ACM needs 'a single self-contained file'!
%
%APPENDICES are optional
%\balancecolumns
\end{document}
