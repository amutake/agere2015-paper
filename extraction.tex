\section{Extraction}
\label{sec:extraction}

Extraction is a Coq feature which enables to convert Coq programs to the programs of other languages.
Normal Coq can extract programs to OCaml, Haskell, and Scheme.
If we want to extract to other languages or use custom extraction algorithm, we have to implement it as plugins or patches.
Actario has custom extraction mechanism for the programs using Actario.
It can extract to Erlang.
It is not proven that the extraction mechanism does not change the meanings of Actario programs and Erlang programs.
In Actario, \lstinline|ActorExtraction| command is defined for extracting actor systems.
It is used like traditional \lstinline|Extraction| command.

\subsection{Data Types}

Values of algebraic data types are extracted to a tuple with the label.
Value constructor is extracted to a label, and arguments are extracted to the second and the following elements of the tuple.
Figure \ref{coq:adt} is an example of extraction of the natural number type.

\begin{figure}[t]
  \begin{lstlisting}
    (* Inductive nat :=  *)
    (* | O : nat         *)
    (* | S : nat -> nat. *)

    O (* => {o} *)
    S (S (S O)) (* => {s, {s, {s, o}}} *)
  \end{lstlisting}
  \caption{example of extraction of algebraic data types}\label{coq:adt}
\end{figure}

However, actions of actors, for example, \textsf{send}, \textsf{new}, \textsf{self}, \textsf{become} and \texttt{behavior} are implemented as value constructor of \texttt{actions} and \texttt{behavior} type
We handle these constructors as special to generate the corresponding syntax of Erlang.

%% ただし、アクターモデル特有である動作、例えば send, new, self, become, receive は、Actario では\texttt{actions}, \texttt{behavior} 型の値コンストラクタとして実装されているので、これだとただのタプルのデータとして抽出されてしまう。
%% このようなアクションは Erlang では構文または関数となっているため、このようなアクションだけを特別扱いして、対応する Erlang のものに変換する必要がある。

For example, Actario code shown in figure \ref{coq:extractionex} is extracted to Erlang code shown in figure \ref{erl:extractionex}.

\begin{figure}[t]
  \begin{lstlisting}
CoFixpoint behvA :=
  receive (fun msg =>
    match msg with
      | name_msg sender =>
        me <- self;
        sender ! name_msg me;
        become behvA
      | _ =>
        child <- new behvB;
        child ! msg;
        become behvA
    end)
  \end{lstlisting}
  \caption{Extraction example: Actario code}\label{coq:extractionex}
\end{figure}

\begin{figure}[t]
  \begin{lstlisting}[language=erlang]
behvA() ->
  receive Msg -> case Msg of
    {name_msg, Sender} ->
      Me = self(),
      Sender ! {name_msg, Me},
      behvA()
    _ ->
      Child = spawn(fun() -> behvB() end),
      Child ! Msg
      behvA()
  end.
  \end{lstlisting}
  \caption{Extraction example: Erlang code}\label{erl:extractionex}
\end{figure}

\subsection{Name}
In Actario, a programmer does not make actor names from constructors, so that all of actor names are in variables.
Therefore, all of actor names in extracted code are variables.
These variables are bound by values of name type in Actario, but in Erlang, these variables are bound by process ids.
%% 名前はプログラマが自分でコンストラクタから作るということはしないので、すべてが変数に格納されている。
%% なので Erlang に変換すると変数になっている。
%% Actario で書いたプログラムの name 型の変数には Actario での name 型の値に束縛されているが、Erlang に変換したあとのプログラムの対応する変数は Erlang のプロセスIDに束縛されている。

\subsection{Execution}
The program extracted by Actario is impossible to execute by itself.
So Actario's programmers have to write executor to execute the extracted Actor system in Erlang.
For example, we consider factorial system described in section \ref{sec:overview}.

\begin{lstlisting}
Definition factorial_system (n : nat) (parent : name) : config :=
  init "factorial" (
         x <- new factorial_behv;
         x ! tuple_msg (nat_msg n) (name_msg parent);
         become empty_behv
       ).
\end{lstlisting}

\lstinline|factorial_system| is extracted to the following Erlang code.

\begin{lstlisting}[language=erlang]
factorial_system(N, Parent) ->
    X = spawn(fun() ->
        factorial_behv()
    end),
    X ! {tuple_msg, {nat_msg, N}, {name_msg, Parent}},
    empty_behv().
\end{lstlisting}

To execute this, we have to write executor like the following.
\lstinline[language=erlang]|nat2int| and \lstinline[language=erlang]|int2nat| are auxiliary functions for converting Coq's natural number and Erlang's integer.

\begin{lstlisting}[language=erlang]
-module(fact_main).
-export([fact/1]).

fact(N) ->
    _ = spawn(factorial, factorial_system, [int2nat(N), self()]),
    receive
        {nat_msg, Result} ->
            io:fwrite("fact(~w) = ~w~n", [N, nat2int(Result)]);
        _ ->
            io:fwrite("error~n")
    end.

nat2int({o}) -> 0;
nat2int({s, N}) -> nat2int(N) + 1.

int2nat(0) -> {o};
int2nat(N) when N > 0 -> {s, int2nat(N - 1)};
int2nat(_) -> {o}.
\end{lstlisting}
